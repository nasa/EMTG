\documentclass[11pt]{article}

%% PACKAGES
\usepackage{graphicx}
\usepackage{verbatim}
\usepackage{url}
\usepackage[printonlyused]{acronym}
\usepackage[ruled]{algorithm}
\usepackage{amsmath,amssymb,amsfonts,amsthm}
\usepackage{overpic}
\usepackage{calc}
\usepackage{color}
%\usepackage{times}
%\usepackage{ragged2e}
 \usepackage[margin=1.0in]{geometry}
\usepackage[colorlinks=false]{hyperref}
\usepackage{textcomp}
\usepackage{cite}
\usepackage{mdwlist}
\usepackage{subfiles}
\usepackage{enumitem}
\usepackage{calc}
\usepackage{array}
\usepackage{units}
\usepackage{arydshln,leftidx,mathtools}
\usepackage[caption=false,font=footnotesize]{subfig}
\usepackage{relsize}
\usepackage{float}
\usepackage{makecell}

\usepackage{algorithm}
\usepackage[noend]{algpseudocode}

\makeatletter
\let\@tmp\@xfloat
\usepackage{fixltx2e}
\let\@xfloat\@tmp
\makeatother

\usepackage[subfigure]{tocloft}
\usepackage[singlespacing]{setspace}
%\usepackage[nodisplayskipstretch]{setspace}
%\setstretch{1.0}

%\renewcommand\cftsecafterpnum{\vskip\baselineskip}
%\renewcommand\cftsubsecafterpnum{\vskip\baselineskip}
%\renewcommand\cftsubsubsecafterpnum{\vskip\baselineskip}

%\usepackage{mathtools}
%\usepackage[framed]{mcode}

\usepackage{pgfplots}

\usepackage{cancel}

\usepackage{tikz}
\usetikzlibrary{calc,patterns,decorations.pathmorphing,decorations.markings,fit,backgrounds}

\usepackage[strict]{changepage} %use to manually place figs/tables to get them within the margins

\makeatletter
\g@addto@macro\normalsize{%
  \setlength\abovedisplayskip{0.25pt}
  \setlength\belowdisplayskip{0.25pt}
  \setlength\abovedisplayshortskip{0.25pt}
  \setlength\belowdisplayshortskip{0.25pt}
}
\makeatother



\setlength{\parskip}{\baselineskip}

%% GRAPHICS PATH
\graphicspath{{../../shared_latex_inputs/images}{../../shared_latex_inputs/graphs}}

%% TODO
\newcommand{\todo}[1]{\vspace{5 mm}\par \noindent \framebox{\begin{minipage}[c]{0.98 \columnwidth} \ttfamily\flushleft \textcolor{red}{#1}\end{minipage}}\vspace{5 mm}\par}

%% MACROS
\providecommand{\abs}[1]{\lvert#1\rvert}
\providecommand{\norm}[1]{\lVert#1\rVert}
\providecommand{\dualnorm}[1]{\norm{#1}_\ast}
\providecommand{\set}[1]{\lbrace\,#1\,\rbrace}
\providecommand{\cset}[2]{\lbrace\,{#1}\nobreak\mid\nobreak{#2}\,\rbrace}
\providecommand{\onevect}{\mathbf{1}}
\providecommand{\zerovect}{\mathbf{0}}
\providecommand{\field}[1]{\mathbb{#1}}
\providecommand{\C}{\field{C}}
\providecommand{\R}{\field{R}}
\providecommand{\polar}{\triangle}
\providecommand{\Cspace}{\mathcal{Q}}
\providecommand{\Fspace}{\mathcal{F}}
\providecommand{\free}{\text{\{}\mathsf{free}\text{\}}}
\providecommand{\iff}{\Leftrightarrow}
\providecommand{\qstart}{q_\text{initial}}
\providecommand{\qgoal}{q_\text{final}}
\providecommand{\contact}[1]{\Cspace_{#1}}
\providecommand{\feasible}[1]{\Fspace_{#1}}
\providecommand{\prob}[2]{p(#1|#2)}
\providecommand{\prior}[1]{p(#1)}
\providecommand{\Prob}[2]{P(#1|#2)}
\providecommand{\Prior}[1]{P(#1)}
\providecommand{\parenth}[1] {\left(#1\right)}
\providecommand{\braces}[1] {\left\{#1\right\}}
\providecommand{\micron}{\hbox{\textmu m}}

%% MATH FUNCTION NAMES
\DeclareMathOperator{\conv}{conv}
\DeclareMathOperator{\cone}{cone}
\DeclareMathOperator{\homog}{homog}
\DeclareMathOperator{\domain}{dom}
\DeclareMathOperator{\range}{range}
\DeclareMathOperator{\argmax}{arg\,max}
\DeclareMathOperator{\argmin}{arg\,min}
\DeclareMathOperator{\area}{area}
\DeclareMathOperator{\sign}{sign}
\DeclareMathOperator{\mathspan}{span}
\DeclareMathOperator{\sn}{sn}
\DeclareMathOperator{\cn}{cn}
\DeclareMathOperator{\dn}{dn}
\DeclareMathOperator*{\minimize}{minimize}

\DeclareMathOperator{\atan2}{atan2}

\newtheorem{theorem}{Theorem}
\newtheorem{lemma}[theorem]{Lemma}

%\setlength{\RaggedRightParindent}{2em}
%\setlength{\RaggedRightRightskip}{0pt plus 3em}
%\pagestyle{empty}




\title{{\Huge EMTGv9 Constraint Scripting}}
\vspace{0.5cm}
\author
{
	Donald H. Ellison \thanks{Aerospace Engineer, NASA Goddard Space Flight Center, Flight Dynamics and Mission Design Branch Code 595},
	Jacob A. Englander \thanks{Aerospace Engineer, NASA Goddard Space Flight Center, Flight Dynamics and Mission Design Branch Code 595},
	Ryo Nakamura \thanks{Aerospace Engineer, NASA Goddard Space Flight Center, Flight Dynamics and Mission Design Branch Code 595, detailed from JAXA},
	Noble Hatten \thanks{Aerospace Engineer, NASA Goddard Space Flight Center, Flight Dynamics and Mission Design Branch Code 595}
}
\vspace{0.5cm}

\date{}

\begin{document}

\begin{titlepage}
\maketitle
%\thispagestyle{empty}
\begin{table}[H]
	\centering
	\begin{tabular}{|l|l|l|}
		\hline
		\textbf{Revision Date} & \textbf{Author} & \textbf{Description of Change} \\ \hline
		\date{November 20, 2018} & Donald Ellison & Listing of current implemented constraints \\
		\hline
		\date{January 16, 2019} & Jacob Englander & \makecell[l]{Updated to reflect new organization of .emtgopt file.\\Added overview section.} \\
		\hline
		\date{December 21, 2018} & Jacob Englander & \makecell[l]{Added phase distance constraint, boundary distance constraint,\\ MGAnDSMs maneuver constraints, PSFB thrust angle constraint,\\ and PSFB duty cycle control} \\
		\hline
		\date{January 16, 2018} & Jacob Englander & \makecell[l]{Added introductory material and instructions for how to use the \\scripted constraints with the new modular options file format.} \\
		\hline
		\date{January 23, 2018} & Jacob Englander & \makecell[l]{Added RBP, RPB, RRP, and RPR angle constraint documentation.} \\
		\hline
		\date{February 5, 2019} & Jacob Englander & \makecell[l]{Updated to new options file spec, commentable constraints,\\and BPT angle with constant bounds.} \\
		\hline
		\date{February 11, 2019} & Jacob Englander & \makecell[l]{Added BCF latitude and longitude boundary constraints. Changed\\ parallel shooting constraint text to recognize that PSFB is no longer\\ the only parallel shooting phase type.}\\
		\date{February 14, 2019} & Jacob Englander & \makecell[l]{Added velocity declination constraint.}\\
		\hline
		\date{February 28, 2019} & Jacob Englander & \makecell[l]{Added vertical flight path angle and velocity magnitude constraints.}\\
		\hline
		\date{October 10, 2019} & Ryo Nakamura & \makecell[l]{Added spinning body-relative velocity magnitude constraint.}\\
		\hline
    \date{October 23, 2019} & Ryo Nakamura & \makecell[l]{Added  latitude, detic altitude, spherical velocity azimuth,\\ relative velocity azimuth, and relative velocity HFPA constraint.}\\
		\hline
    \date{February 12, 2021} & Noble Hatten & \makecell[l]{Created ``How-To'' section for creating a new scripted constraint.}\\
    \hline
    \date{July 29, 2021} & Noble Hatten & \makecell[l]{Added two-body rotating frame state boundary constraint.}\\
		\hline
	\date{July 30, 2021} & Noble Hatten & \makecell[l]{Added new content to how-to section.}\\
	\hline
	\date{September 8, 2021} & Noble Hatten & \makecell[l]{Added PyEMTG interface description.}\\
	\hline
	\end{tabular}
\end{table}
\end{titlepage}



\newpage
\tableofcontents
\thispagestyle{empty}
\newpage

\clearpage
\setcounter{page}{1}




\section*{List of Acronyms}
\begin{acronym}
%To define the acronym and include it in the list of acronyms: \acro{acronym}{definition}
%To define the acronym and exclude it from the list of acronyms:  \acro{acronym}{definition}
%
%\ac{acronym} Expand and identify the acronym the first time; use only the acronym thereafter
%\acf{acronym} Use the full name of the acronym.
%\acs{acronym} Use the acronym, even before the first corresponding \ac command
%\acl{acronym}  Expand the acronym without using the acronym itself.
%
%

\acro{ACO}{Ant Colony Optimization}
\acro{AD}{Automatic Differentiation}
\acro{ADL}{Architecture Design Laboratory}
\acro{AES}{Advanced Exploration Systems}
\acro{AGA}{aerogravity assist}
\acro{ALARA}{As Low As Reasonably Achievable}
\acro{API}{application programming interface}
\acro{BB}{branch and bound}
\acro{BVP}{Boundary Value Problem}
\acro{CATO}{Computer Algorithm for Trajectory Optimization}
\acro{CL}{confidence level}
\acro{CONOPS}{concept of operations}
\acro{COV}{Calculus of Variations}
\acro{D/AV}{Descent/Ascent Vehicle}
\acro{DE}{Differential Evolution}
\acro{DLA}{Declination of Launch Asymptote}
\acro{DPTRAJ/ODP}{Double Precision Trajectory and Orbit Determination Program}
\acro{DSH}{Deep Space Habitat}
\acro{DSN}{Deep Space Network}
\acro{DSMPGA}{Dynamic-Size Multiple Population Genetic Algorithm}
\acro{EB}{Evolutionary Branching}
\acro{ECLSS}{environmental control and life support system}
\acro{ELV}{expendable launch vehicle}
\acro{EMME}{Earth to Mars, Mars to Earth}
\acro{EMMVE}{Earth to Mars, Mars to Venus to Earth}
\acro{EMTG}{Evolutionary Mission Trajectory Generator}
\acro{EVMME}{Earth to Venus to Mars, Mars to Earth}
\acro{EVMMVE}{Earth to Venus to Mars, Mars to Venus to Earth}
\acro{ERRV}{Earth Return Re-entry Vehicle}
\acro{FISO}{Future In-Space Operations}
\acro{FMT}{Fast Mars Transfer}
\acro{GASP}{Gravity Assist Space Pruning}
\acro{GCR}{galactic cosmic radiation}
\acro{GRASP}{Greedy Randomized Adaptive Search Procedure}
\acro{GSFC}{Goddard Space Flight Center}
\acro{GTOC}{Global Trajectory Optimization Competition}
\acro{GTOP}{Global Trajectory Optimization Problem}
\acro{HAT}{Human Architecture Team}
\acro{HGGA}{Hidden Genes Genetic Algorithm}
\acro{IMLEO}{Initial Mass in \acl{LEO}}
\acro{IPOPT}{Interior Point OPTimizer}
\acro{ISS}{International Space Station}
\acro{JHUAPL}{Johns Hopkins University Applied Physics Laboratory}
\acro{JSC}{Johnson Space Center}
\acro{KKT}{Karush-Kuhn-Tucker}
\acro{LEO}{Low Earth Orbit}
\acro{LRTS}{lazy race tree search}
\acro{MONTE}{Mission analysis, Operations, and Navigation Toolkit Environment}
\acro{MCTS}{Monte Carlo tree search}
\acro{MGA}{Multiple Gravity Assist}
\acro{MIRAGE}{Multiple Interferometric Ranging Analysis using GPS Ensemble}
\acro{MOGA}{Multi-Objective Genetic Algorithm}
\acro{MOSES}{Multiple Orbit Satellite Encounter Software}
\acro{MPI}{message passing interface}
\acro{MPLM}{Multi-Purpose Logistics Module}
\acro{MSFC}{Marshall Space Flight Center}
\acro{NELLS}{NASA Exhaustive Lambert Lattice Search}
\acro{NSGA}{Non-Dominated Sorting Genetic Algorithm}
\acro{NSGA-II}{Non-Dominated Sorting Genetic Algorithm II}
\acro{NHATS}{Near-Earth Object Human Space Flight Accessible Targets Study}
\acro{NTP}{Nuclear Thermal Propulsion}
\acro{OD}{orbit determination}
\acro{OOS}{On-Orbit Staging}
\acro{PCC}{Pork Chop Contour}
\acro{PEL}{permissible exposure limits}
\acro{PLATO}{PLAnetary Trajectory Optimization}
\acro{REID}{risk of exposure-induced death}
\acro{RTBP}{Restricted Three Body Problem}
\acro{SA}{Simulated Annealing}
\acro{SLS}{Space Launch System}
\acro{SNOPT}{Sparse Nonlinear OPTimizer}
\acro{SOI}{sphere of influence}
\acro{SPE}{solar particle events}
\acro{SQP}{sequential quadratic programming}
\acro{SRAG}{Space Radiation Analysis Group}
\acro{TEI}{Trans-Earth Injection}
\acro{TOF}{time of flight}
\acro{TPBVP}{Two Point Boundary Value Problem}
\acro{TMI}{Trans-Mars Injection}
\acro{VARITOP}{Variational calculus Trajectory Optimization Program}
\acro{VILM}{v-infinity leveraging maneuver}
\acro{MOI}{Mar Orbit Injection}
\acro{PCM}{Pressurized Cargo Module}
\acro{STS}{Space Transportation System}
\acro{EDS}{Earth Departure Stage}
\acro{NEO}{near-Earth asteroid}
\acro{IDC}{Integrated Design Center}
\acro{SEP}{solar-electric propulsion}
\acro{SRP}{solar radiation pressure}
\acro{NEP}{nuclear-electric propulsion}
\acro{REP}{radioisotope-electric propulsion}
\acro{DRM}{Design Reference Missions}

\acro{ASCII}{American Standard Code for Information Interchange}
\acro{AU}{Astronomical Unit}
\acro{BWG}{Beam Waveguides}
\acro{CCB}{Configuration Control Board}
\acro{CMO}{Configuration Management Office}
\acro{CODATA}{Committee on Data for Science and Technology}
\acro{DEEVE}{Dynamically Equivalent Equal Volume Ellipsoid}
\acro{DRA}{Design Reference Asteroid}
\acro{EME2000}{Earth Centered, Earth Mean Equator and Equinox of J2000 (Coordinate Frame)}
\acro{EOP}{Earth Orientation Parameters}
\acro{ET}{Ephemeris Time}
\acro{FDS}{Flight Dynamics System}
\acro{FTP}{File Transfer Protocol}
\acro{GSFC}{Goddard Space Flight Center}
\acro{PI}{Principal Investigator}
\acro{HEF}{High Efficiency}
\acro{IAG}{International Association of Geodesy}
\acro{IAU}{International Astronomical Union}
\acro{IERS}{International Earth Rotation and Reference Systems Service}
\acro{ICRF}{International Celestial Reference Frame}
\acro{ITRF}{International Terrestrial Reference System}
\acro{IOM}{Interoffice Memorandum}
\acro{JD}{Julian Date}
\acro{JPL}{Jet Propulsion Laboratory}
\acro{LM}{Lockheed Martin}
%\acro{LP150Q}{}
%\acros{LP100K}{}
\acro{MAVEN}{Mars Atmosphere and Volatile EvolutioN}
\acro{MJD}{Modified Julian Date}
\acro{MOID}{Minimum Orbit Intersection Distance}
\acro{MPC}{Minor Planet Center}
\acro{NASA}{National Aeronautics and Space Administration}
\acro{NDOSL}{\ac{NASA} Directory of Station Locations}
\acro{NEA}{near-Earth asteroid}
\acro{NEO}{near-Earth object}
\acro{NIO}{Nav IO}
\acro{OSIRIS-REx}{Origins Spectral Interpretation Resource Identification Security-Regolith Explorer}
\acro{PHA}{Potentially Hazardous Asteroid}
\acro{PHO}{Potentially Hazardous Object}
\acro{SBDB}{Small-Body Database}
\acro{SI}{International System of Units}
\acro{SPICE}{Spacecraft Planet Instrument Camera-matrix Events}
\acro{SPK}{SPICE Kernel}
\acro{SRC}{Sample Return Capsule}
\acro{SSD}{Solar System Dynamics}
\acro{STK}{Systems Tool Kit}
\acro{TAI}{International Atomic Time}
\acro{TBD}{To Be Determined}
\acro{TBR}{To Be Reviewed}
\acro{TCB}{Barycentric Coordinate Time}
\acro{TDB}{Temps Dynamiques Barycentrique, Barycentric Dynamical Time}
\acro{TDT}{Terrestrial Dynamical Time}
\acro{TT}{Terrestrial Time}
\acro{URL}{Uniform Resource Locator}
\acro{UT}{Universal Time}
\acro{UT1}{Universal Time Corrected for Polar Motion}
\acro{UTC}{Coordinated Universal Time}
\acro{USNO}{U. S. Naval Observatory}
\acro{YORP}{Yarkovsky-O'Keefe-Radzievskii-Paddack}

\acro{NLP}{nonlinear program}
\acro{MBH}{monotonic basin hopping}
\acro{MBH-C}{monotonic basin hopping with Cauchy hops}
\acro{FBS}{forward-backward shooting}
\acro{MGALT}{multiple gravity assist with low-thrust}
\acro{MGALTS}{multiple gravity assist with low-thrust using a Sundman transformation}
\acro{MGA-1DSM}{Multiple Gravity Assist with One Deep Space Maneuver}
\acro{MGAnDSMs}{multiple gravity assist with n deep-space maneuvers using shooting}
\acro{PSFB}{Parallel Shooting with Finite-Burn}
\acro{PSBI}{Parallel Shooting with Bounded Impulses}
\acro{FBLT}{finite-burn low-thrust}
\acro{ESA}{European Space Agency}
\acro{ACT}{Advanced Concepts Team}
\acro{IRAD}{independent research and development}
%\acro{Isp}[$\text{I}_{sp}$]{specific impulse}
\acro{GA}{genetic algorithm}
\acro{GALLOP}{ Gravity Assisted Low-thrust Local Optimization Program}
\acro{MALTO}{Mission Analysis Low-Thrust Optimization}
\acro{PaGMO}{Parallel Global Multiobjective Optimizer}
\acro{FRA}{feasible region analysis}
\acro{CP}{conditional penalty}
\acro{HOC}{hybrid optimal control}
\acro{HOCP}{hybrid optimal control problem}
\acro{PSO}{particle swarm optimization}
\acro{SEPTOP}{Solar Electric Propulsion Trajectory Optimization Program}
\acro{STOUR}{Satellite Tour Design Program}
\acro{STOUR-LTGA}{Satellite Tour Design Program - Low Thrust, Gravity Assist}
\acro{PaGMO}{Parallel Global Multiobjective Optimizer}
\acro{SDC}{static/dynamic control}
\acro{DDP}{Differential Dynamic Programming}
\acro{HDDP}{Hybrid Differential Dynamic Programming}
\acro{ACT}{Advanced Concepts Team}
\acro{GMAT}{General Mission Analysis Toolkit}
\acro{BOL}{beginning of life}
\acro{EOL}{end of life}
\acro{KSC}{Kennedy Space Center}
\acro{VSI}{variable \ac{Isp}}
\acro{RTG}{radioisotope thermal generator}
\acro{ASRG}{advanced Stirling radiosotope generator}
\acro{ARRM}{Asteroid Robotic Redirect Mission}
\acro{AATS}{Alternative Architecture Trade Study}
\acro{PPU}{power processing unit}
\acro{STM}{state transition matrix}
\acro{MTM}{maneuver transition matrix}
\acro{BCI}{body-centered inertial}
\acro{BCF}{body-centered fixed}
\acro{UTTR}{Utah Test and Training Range}
\acro{EPV}{equatorial projection of $\mathbf{v}_\infty$}
\acro{KBO}{Kuiper belt object}
\acro{DSM}{deep-space maneuver}
\acro{BPT}{body-probe-thrust}
\acro{4PL}{four parameter logistic}
\acro{BCF}{body-centered fixed}

\end{acronym}

% --------------------------------------------------------------------------------------------------------------------------
% --------------------------------------------------------------------------------------------------------------------------


\section{Overview}
\label{sec:overview}

EMTGv9 allows the user to specify a wide range of scripted constraints beyond what is available in the GUI. This document describes the three classes of constraints - boundary constraints, maneuver constraints, and phase distance constraints (also known as path constraints).

EMTGv9's .emtgopt input file contains a separate, copyable block for each journey in the mission. At the end of each journey, blocks exist for each of the three classes of constraints as shown below:

\begin{verbatim}
#Maneuver constraint code
#Works for absolute and relative epochs and also magnitudes
BEGIN_MANEUVER_CONSTRAINT_BLOCK
END_MANEUVER_CONSTRAINT_BLOCK


#Boundary constraint code
BEGIN_BOUNDARY_CONSTRAINT_BLOCK
END_BOUNDARY_CONSTRAINT_BLOCK


#Phase distance constraint code
BEGIN_PHASE_DISTANCE_CONSTRAINT_BLOCK
END_PHASE_DISTANCE_CONSTRAINT_BLOCK
\end{verbatim}

Constraints are always prefixed with a tag representing the phase that they constrain. Tags are indexed from zero. For example, constraints on the first phase of the journey are prefixed \texttt{p0}, and constraints on the fifth phase are prefixed \texttt{p4}. The user can also place constraints on the \textit{last} phase of the journey using the prefix \texttt{pEnd}.

Internally, the constraints are stored in the JourneyOptions structure as lists of strings:

\begin{verbatim}
JourneyOptions.ManeuverConstraintDefinitions
JourneyOptions.BoundaryConstraintDefinitions
JourneyOptions.PhaseDistanceConstraintDefinitions
\end{verbatim}

PyEMTG collects the journey-level lists of constraints into ``master'' constraint lists that exist at the mission level as shown below. To do this, call \texttt{MissionOptions.AssembleMasterConstraintVectors}. You can also disassemble the master constraint list back to the journey level via 

 \texttt{MissionOptions.DisassembleMasterDecisionVector}.

\begin{verbatim}
MissionOptions.ManeuverConstraintDefinitions
MissionOptions.BoundaryConstraintDefinitions
MissionOptions.PhaseDistanceConstraintDefinitions
\end{verbatim}

If you wish to edit scripted constraints at the journey level via Python, do the following:

\begin{verbatim}
MissionOptions.DisassembleMasterDecisionVector()
<your constraint edits here>
MissionOptions.AssembleMasterConstraintVectors()
\end{verbatim}

Alternatively, if you wish to edit at the mission level, perform the procedure below. However, you will need to prefix \texttt{jJ} to the constraint names when manipulating the master constraint definition lists, where \texttt{J} is the index of the journey that you wish to edit.

\begin{verbatim}
MissionOptions.AssembleMasterConstraintVectors()
<your constraint edits here>
MissionOptions.DisassembleMasterDecisionVector()
\end{verbatim}

Remember to call \texttt{MissionOptions.DisassembleMasterDecisionVector} and \\ \texttt{MissionOptions.AssembleMasterDecisionVector} as appropriate so that the mission and journey-level constraint definition lists remain in sync. In a future update, we will find a way to make this happen automatically.

The user may comment out a constraint with the \texttt{\#} character. The constraint will remain in the script through ingest and rewrite in PyEMTG and EMTG, but will not be used in optimization.

\section{Boundary Constraints}
\label{sec:boundaryConstraints}

Boundary constraints may be applied to any boundary condition class. In order to apply the constraint to the left boundary of a particular phase/journey, the constraint should be tagged as a departure constraint (e.g. p2\_departure\_orbitperiod\_1yr\_1.1yr). Likewise, for a right boundary, it would be tagged as ``arrival''. 

Moving forward, the ability to apply the constrain infinitesimally before or after the boundary of choice is also being added by appending a ``-'' or ``+'' to the boundary name, yielding four possibilities: ``departure-'', ``departure+'', ``arrival-'', and ``arrival+''. The difference between ``-'' and ``+'' is that the ``-'' version does not take into account events that happen instantaneously at the boundary (like a $\Delta v$ or a mass drop), while the ``+'' version does. So far, the state in two-body rotating frame constraint is the only constraint that supports this capability and syntax. For constraints that do not support this syntax, the user must look at the \texttt{process\_constraint} method of the boundary constraint to determine whether the constraint is evaluate before or after the boundary event.

Boundary constraints are specified in the boundary constraint block at the end of each journey in the .emtgopt file. (See Section~\ref{sec:overview}.)

\subsection{How to Create a New Boundary Constraint}
\label{sec:howToCreateANewBoundaryConstraint}

Creating a new boundary constraint consists of creating a new class derived from \texttt{SpecializedBoundaryConstraintBase} and editing existing code to make EMTG aware of the existence of the new constraint. All required steps are described in this section.

\textbf{The easiest way to create a new specialized boundary constraint is to examine (i.e., copy and paste) an existing boundary constraint to use as a starting point.} As of July 30, 2021, it is recommended to start from the \texttt{BoundaryStateInTwoBodyRotatingFrameConstraint} class because it contains most if not all of the features required to create a new boundary constraint.

\subsubsection{New Files}

A new boundary constraint requires creating a new class derived from \texttt{SpecializedBoundaryConstraintBase}. Thus, a \texttt{.h} and a \texttt{.cpp} file must be created. These files are placed in 

\begin{verbatim}
emtg/src/Mission/Journey/Phase/BoundaryEvents/SpecializedBoundaryConstraints/
\end{verbatim}

In addition, the CMakeLists.txt file in the preceding directory must be edited so that the new header file is added to the list of \texttt{SPECIALIZED\_BOUNDARY\_EVENT\_HEADERS} and the new source file is added to the list of \texttt{SPECIALIZED\_BOUNDARY\_CONSTRAINTS\_SOURCE}. The EMTG cmake file must be reconfigured and regenerated after the new files are added to the CMakeLists.txt file in order for the changes to take effect.


\subsubsection{Structure of the New Class}

Several methods of \texttt{SpecializedBoundaryConstraintBase} must be overridden by the new constraint class:

\begin{itemize}
	\item \texttt{output}: Prints the name and value of the constraint to the .emtg output file. The text is displayed underneath the state and control history output for the journey on which the constraint is applied. 
	\item \texttt{calcbounds}: Parses the text of the constraint definition; sets the upper and lower bounds of the constraint; sets the sparsity pattern of the constraint. Called once per execution, prior to optimization.
	\item \texttt{process\_constraint}: Calculates the value of the constraint and, optionally, its Jacobian.
\end{itemize}

In addition, it may be useful to add code to the constraint's constructor method. E.g., the user may wish to set a reference frame.

It is common practice to create class variables for a boundary constraint class. In particular, a class variable for the value being constrained can be used in the \texttt{output} method to print the value.

\subsubsection{The Constructor Method}

The constructor method is constraint-specific. There are two common argument lists for constructors: one that contains the \texttt{constraint\_reference\_frame} argument and one that does not. The \texttt{constraint\_reference\_frame} argument is required if the user is allowed to specify the reference frame in which the constraint is applied. An example is \texttt{InclinationConstraint}, which may be found in the \texttt{OrbitElementsConstraints} directory. The \texttt{constraint\_reference\_frame} argument is not required if the user is not allowed to specify the reference frame in which the constraint is applied. An example is \texttt{SemimajorAxisConstraint}, which may be found in the \texttt{OrbitElementsConstraints} directory.

In some cases, the contents of the constructor method is empty. Examples of tasks that are commonly performed in a constructor include:

\begin{itemize}
	\item Assigning the \texttt{constraint\_reference\_frame} argument to a class variable, if relevant.
	\item Setting the dimensions of any \texttt{Matrix} class variables.
	\item Switching on the computation of orbit elements at the boundary event. See Section~\ref{sec:CreatingConstraintTipsAndTricks} for more details.
\end{itemize}

\subsubsection{The \texttt{output} Method}

The \texttt{output} method is extremely simple and consists of printing the value of the constraint and, if desirable, basic associated metadata (e.g., reference frame in which the constraint was computed), to the .emtg output file. The text is displayed underneath the state and control history output for the journey on which the constraint is applied.  

\subsubsection{The \texttt{calcbounds} Method}

The first task of \texttt{calcbounds} is to parse the text of \texttt{this->constraintDefinition}, which contains the name of the constraint, when the constraint should be applied, the upper and lower bounds of the constraint, and any additional information required to specify the constraint. The different parts of the constraint definition are separated by underscores, and the \texttt{boost::split} function is used to extract the individual parts.

The text description of the constraint is set with

\texttt{this->Fdescriptions->push\_back(prefix + "<description>")}

where \texttt{<description>} is a short description of the constraint. In the \texttt{SpecializedBoundaryConstraintBase} \texttt{setup\_calcbounds} method, \texttt{prefix} is set to \texttt{this->name + ": "}. Note: Do not put a comma in any of the text that will be part of \texttt{Fdescriptions} because it will mess up csv readers.

The lower bound of the constraint is set with

\texttt{this->Flowerbounds->push\_back(<lowerBound>)}

The upper bound is set with

\texttt{this->Fupperbounds->push\_back(<upperBound>)}

\textbf{Important:} If it is desired to scale the constraint, then the values added to \texttt{Flowerbounds} and \texttt{Fupperbounds} must be scaled appropriately.

The final job of \texttt{calcbounds} is to set the sparsity pattern of the constraint function. This process can be quite confusing, so copy/pasting and making minimal adaptations for a specific constraint from a previous boundary constraint class is highly recommended. This walkthrough assumes that the reader is following along with the code in \texttt{BoundaryStateInTwoBodyRotatingFrameConstraint.cpp}.

The first call when creating the sparsity pattern is

\begin{verbatim}
	std::vector< std::tuple<size_t, size_t, double> >&
	 Derivatives_of_StateAroundEvent = this->myBoundaryEvent->
	get_Derivatives_of_StateBeforeOrAfterEvent(this->getStateAndDerivativesIndex);
\end{verbatim}

\noindent The first element of each returned tuple in \texttt{Derivatives\_of\_StateAroundEvent} is the index of the decision variable in the global decision vector. The second element of each returned tuple in \texttt{Derivatives\_of\_StateAroundEvent} is the local position of the decision variable in the state vector. Note that the position/velocity of the state vector is always in ICRF Cartesian coordinates, regardless of the frame and state representation used to encode the decision variables. The third element of each returned tuple in \texttt{Derivatives\_of\_StateAroundEvent} is the actual derivative value. When \texttt{calcbounds} is called, no value has been set, but that is OK \textemdash we do not need it for setting the sparsity pattern. We can think of the elements of \texttt{Derivatives\_of\_StateAroundEvent} as: the derivative of the second element with respect to the first element is equal to the third element.

We do this for derivatives with respect to time, as well:

\begin{verbatim}
std::vector< std::tuple<size_t, size_t, double> >& 
Derivatives_of_StateAroundEvent_wrt_Timethis->myBoundaryEvent->
get_Derivatives_of_StateBeforeOrAfterEvent_wrt_Time(this->getStateAndDerivativesIndex);
\end{verbatim}

We then loop through state indexes. In the two-body rotating frame constraint example, this is done by looping from 0 to 2, and handling velocity separately from position within the same loop. An alternative formulation could loop from 0 to 5.

Within the loop, the first action handles the derivatives with respect to position. We loop through each element of the \texttt{Derivatives\_of\_StateAroundEvent} vector. For each element, we check if second element of that tuple is equal to our current state index (the loop variable of our outer loop). In other words, does our current index correspond to differentiating an element of the position state? If so, we wish to add a sparsity entry for it because our constraint depends on all elements of the position vector. So, we call

\begin{verbatim}
this->create_sparsity_entry(this->Fdescriptions->size() - 1,
std::get<0>(Derivatives_of_StateAroundEvent[dIndex]),
state_Gindex_constraint_position_wrt_StateAroundEvent_variables);
\end{verbatim}

\noindent The first argument to this call is the index of the current constraint in the overall vector of constraints. The second argument is the position of the independent variable currently under consideration in the overall vector of decision variables. The third argument is an output: the vector of global Jacobian entry indexex, which is a single-dimension vector because it is an unrolled vector.

After looping through the elements of the \texttt{Derivatives\_of\_StateAroundEvent} vector, we call

\begin{verbatim}
this->dIndex_constraint_position_wrt_StateAroundEvent.push_back
(state_dIndex_constraint_position_wrt_StateAroundEvent);

this->Gindex_constraint_position_wrt_StateAroundEvent_variables.push_back
(state_Gindex_constraint_position_wrt_StateAroundEvent_variables);
\end{verbatim}

\noindent These calls basically put what were originally local variables into class variables that can be accessed from later calls to \texttt{process\_constraint}.

We then do a similar loop to create sparsity entries for ``implicit'' dependencies of state elements on time variables. These exist because we are using a multiple-shooting method. The state and time loops are separate because we can have non-zero partial derivatives with respect to time variables from other phases in a multiple-shooting algorithm, whereas we will never have non-zero partial derivatives with respect to state variables from other phases. For example, the position state at a boundary depends on the launch epoch and the times of flight of all previous phases.

After handling the dependencies of position on state and time decision variables, two more loops are performed to do the same thing for velocity. In the two-body rotating frame example, the constraint only depends on velocity if a velocity state is being constrained. As a result, the velocity sparsity pattern creation loops are in an \texttt{if} block.

After completing the loop through state indexes, we still need to handle \emph{explicit} time dependencies, if they exist. An explicit time dependence exists if the constraint value depends on an ephemeris lookup or a time-dependent rotation matrix. If explicit time dependencies exist, we call

\begin{verbatim}
std::vector<size_t> timeVariables = this->myBoundaryEvent->get_Xindices_EventRightEpoch();
\end{verbatim}

\noindent and then loop through \texttt{timeVariables}, creating sparsity entries. Note that this workflow only works because EMTG constructs constraints sequentially in the order in which time marches forward. At the time \texttt{calcbounds} is called for a constraint, subsequent journeys haven't been constructed yet.

The \texttt{calcbounds} of each specialized constraint is only called once in each EMTG execution, prior to optimization.

\subsubsection{The \texttt{process\_constraint} Method}

The \texttt{process\_constraint} method is called whenever EMTG needs to evaluate the value of the constraint and/or its Jacobian.

The arguments are:

\begin{itemize}
	\item \texttt{X}: The decision vector. Often, \texttt{X} is not used directly because more useful routines exist to do things like extract the state at the boundary.
	\item \texttt{Xindex}: This argument is not actually used, and, confusingly, is overridden locally in the \texttt{process\_constraint} method of many constraints.
	\item \texttt{F}: Vector of constraints. The value of the constraint is added to this vector.
	\item \texttt{Findex}: A locator of a constraint within the vector of constraints \texttt{F}.
	\item \texttt{G}: The Jacobian of the constraint with respect to all decision variables. Must be populated if \texttt{needG} is true.
	\item \texttt{needG}: If true, the Jacobian of the constraint must be calculated. If false, it need not be.
\end{itemize}

The value of the constraint must be calculated. Recall that the value of the constraint is generally assigned to a class member so that it is accessible from the \texttt{output} method. Methods exist for doing things like obtaining the state of the spacecraft at the point of constraint application. Once the value is calculated, it is added to the vector of constraints with

\texttt{F[Findex++] = <value>}

If the Jacobian is required, it must be calculated. Frequently, the dependencies of a constraint are some subset of position, velocity, and time at, infinitesimally before, or infinitesimally after a boundary event.\footnote{``Infinitesimally before/after'' means before/after an instantaneous boundary event, such as a $\Delta v$, takes place.} A new specialized boundary constraint is responsible for calculating these ``first-level'' derivatives, and, then, using the chain rule to achieve the derivatives of the constraint with respect to the decision variables themselves. The derivatives of the state vector with respect to the decision variables at the boundary event are accessible to facilitate this procedure.

If the Jacobian is required, it must be placed in the correct entries of \texttt{G}. Determining the correct entries of \texttt{G} is similar to the procedure used to set the sparsity pattern in \texttt{calcbounds}. Like for setting the sparsity pattern, this walkthrough assumes that the reader is following along with the code in \texttt{BoundaryStateInTwoBodyRotatingFrameConstraint.cpp}. We follow these steps:

\begin{enumerate}
	\item Prior to setting the relevant elements of \texttt{G}, it is standard practice to set them to zero. We loop through the relevant state indices and the relevant entries of \texttt{this->Gindex\_constraint\_position\_wrt\_StateAroundEvent\_variables} to know which elements to set to zero. We also loop through the explicit time dependence indexes, if relevant.
	\item We then set the derivatives of the constraint with respect to non-time variables affecting the boundary state. We grab the derivatives of the state at the boundary with respect to decision variables because this is the last link in the chain rule chain we need to create. Then, we loop through various indexes to place our derivatives appropriately. The things we need to keep in mind are:
	\begin{itemize}
		\item Our \texttt{Gentry} needs to contain our local derivative entry (constraint with respect to boundary state) multiplied with the derivative of the state at the boundary with respect to the relevant decision variable, which is held in \texttt{std::get<2>(Derivatives\_of\_StateAroundEvent[dIndex])}.
		\item When we add our \texttt{Gentry} to \texttt{G[Gindex]}, the premultiplication by \texttt{this->X\_scale\_factors->operator[](Xindex)} takes care of scaling the decision variable, but it is our responsibility to scale by the scale factor of the constraint, if one was used.
	\end{itemize}
	\item We then set derivatives of the constraint with respect to time variables affecting the boundary state (i.e., the implicit time dependencies). This procedure is extremely similar to that for setting the derivatives with respect to the boundary state itself. The important thing to remember here is that our local part of the chain rule still uses derivatives of our constraint with respect to the boundary state, \emph{not} time. Time dependence comes through the other part of the chain rule: the time derivatives of the boundary state.
	\item If necessary, we then set explicit time dependencies. This is where derivatives of the constraint directly with respect to time are used. There is no additional chain rule beyond that here.
\end{enumerate}

\subsubsection{Updating \texttt{SpecializedBoundaryConstraintFactory}}

In addition to creating a new class derived from \texttt{SpecializedBoundaryConstraintBase}, a new block of code must be added to the \texttt{create\_boundary\_event\_constraint} function in the file \texttt{SpecializedBoundaryConstraintFactory.cpp}. \texttt{create\_boundary\_event\_constraint} parses the text of a specialized boundary constraint and ``decides'' which specialized boundary class should be used to instantiate a new specialized boundary constraint object based on the text. So, to make a new specialized boundary event accessible to users, new \texttt{else if} blocks must be added that look for \emph{uniquely identifiable} text within the text of the new specialized boundary constraint. Frequently, constraint text is organized as (all on one line)

\begin{verbatim}
p<phase number>_<departure or arrival><- or +>_<constraint name>_
<lower bound><lower bound units>_<upper bound><upper bound units>
\end{verbatim}

(See the examples in later sections.) \texttt{create\_boundary\_event\_constraint} splits the entire string at each underscore and puts the results in the \texttt{std::vector ConstraintDefinitionCell}. The constraint name --- or \texttt{ConstraintDefinitionCell[2]} --- is almost always used to identify the constraint in \texttt{create\_boundary\_event\_constraint}. Thus, it is important to give the new constraint a unique name --- and to instruct \texttt{create\_boundary\_event\_constraint} to search for unique text!

Within the new \texttt{else if} block, the only action required is to \texttt{return} a \texttt{new} appropriate specialized boundary constraint object. The code from an existing block may therefore be copied into a new block, with the only required change being the name of the class derived from \texttt{SpecializedBoundaryConstraintBase}.

\textbf{Important note:} All of the \texttt{if/else if} blocks based on \texttt{ConstraintDefinitionCell[2]} are contained within an outer \texttt{if/else if} block based on \texttt{EventDefinition}. \texttt{EventDefinition} can be either ``departure'' or ``arrival''. If the desired behavior is for the new constraint to be applicable at either a departure or arrival event, then an identical \texttt{else if} block must be added to both the \texttt{if} and \texttt{else} blocks of the \texttt{EventDefinition}-based block. Alternatively, if the desired behavior is for the new constraint to be applicable to only departure \emph{or} arrival events, then the new code block should still be added to both sections, but should throw a helpful exception if the user attempts to invoke the constraint incorrectly (i.e., if the user tries to apply a departure-only constraint at an arrival event).

\subsubsection{Updating the Executables}
\label{sec:UpdatingTheExecutables}

For some constraints, it may be necessary to update the contents of the executable files \texttt{EMTG\_v9.cpp} and \texttt{testbed\_driver.cpp}. The reason to update the executables is if the new constraint is allowed to depend on a universe body other than the central body of the journey.\footnote{An example of a constraint that does depend on an additional universe body is the RPR angle constraint because one or both of the reference bodies could be a universe body other than the central body. An example of a constraint that does not depend on an additional universe body is the semimajor axis constraint because the semimajor axis is always calculated relative to the central body of the constraint.} If the constraint is allowed to depend on a universe body other than the central body, then the executables must be updated to create splines of the states of any additional universe bodies whose states are required by the constraint. To accomplish this, a new block of code must be placed in each executable inside the \texttt{for} loop that begins

\begin{verbatim}
//boundary constraint list
//some, but not all, require new splines
for (std::string& constraint : options.Journeys[j].BoundaryConstraintDefinitions)
{
	...
}
\end{verbatim}

The new block of code can be modeled after the existing \texttt{if} blocks inside this loop. For example, for the distance constraint, we have:

\begin{verbatim}
// distance constraint
if (boost::to_lower_copy(ConstraintDefinitionCell[2]).find("distanceconstraint") < 1024)
{
	if (boost::to_lower_copy(ConstraintDefinitionCell[3]) != "cb")
	{
		int bodyIndex = std::stoi(ConstraintDefinitionCell[3]) - 1;

		body_index_array.push_back(bodyIndex);
	}
}
\end{verbatim}

The outer \texttt{if} block determines whether a given boundary constraint is used. The inner \texttt{if} block checks to see if a universe body other than the central body is required to calculate the constraint, determines which universe body is required, and adds that body to the array of bodies to be splined. Note that the contents of the inner \texttt{if} block may change based on the individual constraint.


\subsubsection{Tips and Tricks}
\label{sec:CreatingConstraintTipsAndTricks}

Depending on what the constraint is on, it may be appropriate to scale the constraint. For example, constraints on distances are usually scaled by 1 LU.

It may be appropriate to allow the user to set units on the constraint.

Routines for obtaining existing state information:

\begin{itemize}
	\item \texttt{BoundaryEventBase::get\_state\_before\_or\_after\_event(int before\_or\_after)}
	\begin{itemize}
		\item Returns state before event if \texttt{before\_or\_after} == -1; returns state after event if \texttt{before\_or\_after} == 1.
	\end{itemize}
%	\item \texttt{BoundaryEventBase::get\_state\_before\_event()}
%	\item \texttt{BoundaryEventBase::get\_state\_after\_event()}
\end{itemize}

Routines for obtaining existing derivative information:

\begin{itemize}
	\item \texttt{BoundaryEventBase::get\_Derivatives\_of\_StateBeforeOrAfterEvent(int before\_or\_after)}
	\begin{itemize}
		\item Returns state before event if \texttt{before\_or\_after} == -1; returns state after event if \texttt{before\_or\_after} == 1.
	\end{itemize}
	\item \texttt{BoundaryEventBase::get\_Derivatives\_of\_StateBeforeOrAfterEvent\_wrt\_Time(int before\_or\_after)}
	\begin{itemize}
		\item Returns state before event if \texttt{before\_or\_after} == -1; returns state after event if \texttt{before\_or\_after} == 1.
	\end{itemize}
%	\item \texttt{BoundaryEventBase::get\_Derivatives\_of\_StateBeforeEvent()}
%	\item \texttt{BoundaryEventBase::get\_Derivatives\_of\_StateAfterEvent()}
%	\item \texttt{BoundaryEventBase::get\_Derivatives\_of\_StateBeforeEvent\_wrt\_Time()}
%	\item \texttt{BoundaryEventBase::get\_Derivatives\_of\_StateAfterEvent\_wrt\_Time()}
\end{itemize}

Routines for computing orbit elements:

\begin{itemize}
	\item \texttt{BoundaryEventBase::setComputeOrbitElements()}
	\item \texttt{BoundaryEventBase::add\_orbit\_element\_reference\_frame()}
	\item \texttt{BoundaryEventBase::get\_orbit\_elements\_after\_event()}
	\item \texttt{BoundaryEventBase::get\_orbit\_element\_Jacobian\_after\_event()}
\end{itemize}

\subsubsection{A Note on States and Derivatives Before and After Boundaries}
\label{sec:states_and_derivatives_before_and_after_boundaries}

Section~\ref{sec:CreatingConstraintTipsAndTricks} describes methods for obtaining state and derivative information infinitesimally before or after a boundary event. Some constraints, like the constraint on a state in a two-body rotating frame (Section~\ref{subsec:state_in_two_body_rotating_frame}), allow the user to use a \texttt{-} or \texttt{+} to indicate that they wish to constrain the state infinitesimally before or after a boundary event. However, it is important to realize that, while these methods exist, the backend EMTG data, unfortunately, does not always contain the information the user may want or expect when using these methods. The list below describes what EMTG actually stores for commonly used boundary types and classes. In the list, the following shorthand is used:

\begin{itemize}
	\item $x-$: State obtained by \texttt{get\_state\_before\_event()}.
	\item $x+$: State obtained by \texttt{get\_state\_after\_event()}.
	\item $x_1$: State given in .emtg file for \texttt{Boundary: 1} for a journey.
	\item $x_2$: State given in .emtg file for \texttt{Boundary: 2} for a journey.
	\item $x_{e,1}$ State given in .emtg file in first row of mission events for a journey.
	\item $x_{e,-1}$ State given in .emtg file in last row of mission events for a journey.
	\item $\Delta v_1$, $\Delta v_{-1}$ Values given in \texttt{dV\_x}, \texttt{dV\_y}, and \texttt{dV\_z} columns of the first or last mission event in the .emtg file for a journey.
\end{itemize}

\begin{itemize}
	\item Departure types/classes:
	\begin{itemize}
		\item Type 0: Launch or direct insertion
		\begin{itemize}
			\item Class 0: Ephemeris-pegged
			\begin{itemize}
				\item $x- = x+ = x_{e,1}$
				\item $v_{e,1} = v_1 + \Delta v_1$
				\item $\Delta v_1$ is set by the decision variables for $v_\infty$, RA, and DEC.
				\item As currently implemented, I would consider both $x-$ and $x+$ to actually by $x+$ (i.e., after the launch or direct insertion maneuver).
			\end{itemize}
			\item Class 1: Free point
			\begin{itemize}
				\item $x- = x+ = x_{e,1}$
				\item $v_{e,1} = v_1 + \Delta v_1$
				\item $\Delta v_1$ is set by the decision variables for $v_\infty$, RA, and DEC.
				\item As currently implemented, I would consider both $x-$ and $x+$ to actually by $x+$ (i.e., after the launch or direct insertion maneuver).
			\end{itemize}
			\item Class 3: Periapse
			\begin{itemize}
				\item $x- = x+ = x_{e,1} = x_1$
				\item $\Delta v_1$ is zero.
				\item As currently implemented, I would consider both $x-$ and $x+$ to actually by $x+$ (i.e., after the launch or direct insertion maneuver).
			\end{itemize}
		\end{itemize}
		\item Type 2: Free direct departure
		\begin{itemize}
			\item Class 1: Free point
			\begin{itemize}
				\item $x- = x+ = x_{e,1} = x_1$
				\item $\Delta v_1$ is zero.
				\item It makes sense that all position and velocity states are equal because there is no maneuver.
			\end{itemize}
		\end{itemize}
		\item Type 3: Flyby
		\begin{itemize}
			\item Class 0: Ephemeris-pegged
			\begin{itemize}
				\item $x- = x+ = x_{e,1}$
				\item $v_{e,1} = v_1 + $ the journey's $v_\infty$ decision variables.
				\item As currently implemented, I would consider both $x-$ and $x+$ to actually by $x+$ (i.e., after the velocity change from the flyby has been imparted).
			\end{itemize}
		\end{itemize}
	\end{itemize}
	\item Arrival types/classes:
	\begin{itemize}
		\item Type 0: Insertion into parking orbit with chemical Isp
		\begin{itemize}
			\item Class 0: Ephemeris-pegged
			\begin{itemize}
				\item $x- = x+ = x_{e,-1}$
				\item $v_{e,-1} = v_2 - \Delta v_{-1}$
				\item $\Delta v_{-1} =-v_\infty$ from the decision variables.
				\item It makes sense that $x- = x+$ even though $v_2$ is different because a maneuver is not actually being performed at the boundary. EMTG is just comparing the spacecraft state to the boundary state using $v_\infty$.
			\end{itemize}
		\end{itemize}
		\item Type 1: Rendezvous with chemical maneuver
		\begin{itemize}
			\item Class 1: Free point
			\begin{itemize}
				\item $x+ = x+ = x_{e,-1} = x_2$
				\item $v- = (v+) - \Delta v_{-1}$
				\item This one actually behaves out I would expect it to behave.
			\end{itemize}
		\end{itemize}
		\item Type 2: Intercept with bounded v infinity
		\begin{itemize}
			\item Class 0: Ephemeris-pegged
			\begin{itemize}
				\item $x- = x+ = x_{e,-1}$
				\item $v_{e,-1} = v_2 - \Delta v_{-1}$
				\item $\Delta v_{-1} =-v_\infty$ from the decision variables.
				\item It makes sense that $x- = x+$ even though $v_2$ is different because a maneuver is not actually being performed at the boundary. EMTG is just comparing the spacecraft state to the boundary state using $v_\infty$.
			\end{itemize}
			\item Class 1: Free point
			\begin{itemize}
				\item $x- = x+ = x_{e,-1}$
				\item $v_{e,-1} = v_2 - \Delta v_{-1}$
				\item $\Delta v_{-1} =-v_\infty$ from the decision variables.
				\item It makes sense that $x- = x+$ even though $v_2$ is different because a maneuver is not actually being performed at the boundary. EMTG is just comparing the spacecraft state to the boundary state using $v_\infty$.
			\end{itemize}
			\item Class 2: Ephemeris-referenced
			\begin{itemize}
				\item $x- = x_{e,-1}$
				\item $v_{e,-1} = v_2 - \Delta v_{-1}$, which is derived from the decision variables.
				\item $x+ = x_{e,-1}$ BUT $x+$ is returned to the user relative to the boundary-event body, NOT the central body of the journey.
				\item It makes sense that $x- = x+$, though expressed relative to different central bodies, even though $v_2$ is different because a maneuver is not actually being performed at the boundary. EMTG is just comparing the spacecraft state to the boundary state using $v_\infty$.
			\end{itemize}
			\item Class 3: Periapse
			\begin{itemize}
				\item $x- = x+ = x_{e,-1} = x_2$
				\item It makes sense that $x- = x+$ because a maneuver is not actually being performed at the boundary. EMTG is just comparing the spacecraft state to the boundary state using $v_\infty$.
			\end{itemize}
		\end{itemize}
	\end{itemize}
\end{itemize}

In summary, there are some boundaries at which it would be expected that the before/after states be the same. There are there boundaries at which it would be expected that the before/after states be different, but EMTG returns the same value. This may be changed in the future, but it has not been changed yet.

\subsection{Classical Orbit Element Boundary Constraints}
\label{sec:boundaryCOEconstraints}

Distance-based constraints may be specified in kilometers (km), nautical miles (nmi), \ac{AU} (AU) or universe length units (LU). Angular constraints may be specified in degrees (deg) or radians (rad). Time constraints may be specified in seconds (sec), hours (hr), days (d) or years (yr). The orientation COEs require that frame specification (e.g. ICRF, J2000BCI etc.).

\subsubsection{Semimajor Axis}
\label{sec:boundarySMAconstraint}

\begin{verbatim}
BEGIN_BOUNDARY_CONSTRAINT_BLOCK
pP_arrival_SMA_10000km_12000km
END_BOUNDARY_CONSTRAINT_BLOCK
\end{verbatim}

\subsubsection{Inclination}
\label{sec:boundaryINCconstraint}
\begin{verbatim}
BEGIN_BOUNDARY_CONSTRAINT_BLOCK
pP_departure_INC_40deg_40.1deg_J2000BCI
END_BOUNDARY_CONSTRAINT_BLOCK
\end{verbatim}

\subsubsection{Eccentricity}
\label{sec:boundaryECCconstraint}

\begin{verbatim}
BEGIN_BOUNDARY_CONSTRAINT_BLOCK
pP_arrival_ECC_0.6_0.7
END_BOUNDARY_CONSTRAINT_BLOCK
\end{verbatim}

\subsubsection{Right Ascension of the Ascending Node}
\label{sec:boundaryRAANconstraint}

\begin{verbatim}
BEGIN_BOUNDARY_CONSTRAINT_BLOCK
pP_arrival_RAAN_0.1rad_0.15rad_ICRF
END_BOUNDARY_CONSTRAINT_BLOCK
\end{verbatim}

\subsubsection{Argument of Periapse}
\label{sec:boundaryAOPconstraint}

\begin{verbatim}
BEGIN_BOUNDARY_CONSTRAINT_BLOCK
pP_arrival_AOP_0.1rad_0.15rad_TrueOfDateBCF
END_BOUNDARY_CONSTRAINT_BLOCK
\end{verbatim}

\subsubsection{True Anomaly}
\label{sec:boundaryTAconstraint}

\begin{verbatim}
BEGIN_BOUNDARY_CONSTRAINT_BLOCK
pP_arrival_trueanomaly_10deg_15deg_ICRF
END_BOUNDARY_CONSTRAINT_BLOCK
\end{verbatim}

\subsection{COE Derived Constraints}
\label{sec:boundaryCOEderivedConstraints}

\subsubsection{Orbit Period}
\label{sec:boundaryOrbitPeriod}

For closed orbits (i.e. $a > 0.0$) the spacecraft's orbital period $T$ can be constrained:

\begin{equation}
	T = \sqrt{\frac{a^3}{\mu}}
\end{equation}

\begin{verbatim}
	BEGIN_BOUNDARY_CONSTRAINT_BLOCK
	pP_departure_orbitperiod_5.0d_5.3d
	END_BOUNDARY_CONSTRAINT_BLOCK
\end{verbatim}

In general, it is preferable to specify this constraint as an SMA constraint \ref{sec:boundarySMAconstraint} in the event that the spacecraft state results in $a <= 0.0$ (which can happen during optimizer iteration).

\subsubsection{Periapse Distance}
\label{sec:boundaryOrbitPeriapse}

\begin{verbatim}
BEGIN_BOUNDARY_CONSTRAINT_BLOCK
pP_arrival_periapsedistance_10000nmi_12000nmi
END_BOUNDARY_CONSTRAINT_BLOCK
\end{verbatim}

\subsubsection{Apoapse Distance}
\label{sec:boundaryOrbitApoapse}

\begin{verbatim}
BEGIN_BOUNDARY_CONSTRAINT_BLOCK
pP_arrival_apoapsedistance_10000nmi_12000nmi
END_BOUNDARY_CONSTRAINT_BLOCK
\end{verbatim}

\subsection{Orbital Energy}
\label{sec:orbitalEnergyConstraint}

Specific orbital energy $\epsilon$ can be constrained:

\begin{equation}
	\label{eq:orbital_energy}
	\epsilon = \frac{1}{2}v^2 - \frac{\mu}{r}
\end{equation}

\noindent Currently, orbital energy can only be specified in units of $km^2/s^2$.

\begin{verbatim}
	BEGIN_BOUNDARY_CONSTRAINT_BLOCK
	pP_arrival_orbitalenergy_10.0km2s2_10.3km2s2
	END_BOUNDARY_CONSTRAINT_BLOCK
\end{verbatim}

\subsection{Distance Constraint}
\label{sec:boundaryDistanceConstraint}

The user may constrain the distance between any boundary condition and any body in the Universe file. The user specifies the phase prefix, ``arrival'' or ``departure,'' the body of interest as either ``cb'' or an integer index from the Universe file, and lower and upper bounds. The constraint may be posed in km, AU, or Universe LU.

\begin{verbatim}
BEGIN_BOUNDARY_CONSTRAINT_BLOCK
p0_arrival_distanceconstraint_3_0.0au_1.0au
p0_departure_distanceconstraint_cb_0.0au_3.5au
END_BOUNDARY_CONSTRAINT_BLOCK
\end{verbatim}

\subsection{Angle Constraints}
\label{subsec:boundaryAngleConstraint}

The user may constrain the angle between the sun, the spacecraft, and a reference body. This may be done in terms of the Sun-Body-Probe (SBP) or Sun-Probe-Body (SPB). This set of constraints can be used to represent anything from the phase angle at encounter with an asteroid (Sun-asteroid-Probe) or the Sun-Earth-Probe (SEP) or Sun-Probe-Earth (SPE) angles.

There are four versions of these constraints as detailed below. The first two versions apply only to ephemeris-pegged boundary conditions that encode a $v_\infty$ vector, such as intercept/flyby, chemical rendezvous, and orbit insertion. The last two may be applied to any class of boundary condition.

\begin{enumerate}
	\item \textbf{Ephemeris-pegged Reference-Body-Probe (RBP)}. The user only needs to provide the reference body. The constraint uses the boundary position vector to represent the Body and the sum of the boundary position vector and the spacecraft $v_\infty$ vector to represent the Probe. An example would be Sun-asteroid-Probe angle for an ephemeris-pegged asteroid flyby, where the Sun is the Reference and the asteroid is the Body.
	\item \textbf{Ephemeris-pegged Reference-Probe-Body (RPB)}. The user only needs to provide the reference body. The constraint uses the boundary position vector to represent the Body and the sum of the boundary position vector and the spacecraft $v_\infty$ vector to represent the Probe. An example would be Sun-Probe-asteroid angle for an ephemeris-pegged asteroid flyby, where the Sun is the Reference and the asteroid is the Body.
	\item \textbf{Reference-Reference-Probe (RRP)}. The user specifies two reference bodies and the spacecraft's position vector is used for the Probe. One example is Sun-Earth-Probe angle, where the Sun and the Earth are the two references. Another example is Sun-asteroid-Probe angle for a flyby that is specified with any boundary class other than ephemeris-pegged.
	\item \textbf{Reference-Probe-Reference (RPR)}. The user specifies two reference bodies and the spacecraft's position vector is used for the Probe. One example is Sun-Probe-Earth angle, where the Sun and the Earth are the two references. Another example is Sun-Probe-asteroid angle for a flyby that is specified with any boundary class other than ephemeris-pegged.
\end{enumerate}

\subsubsection{RBP and RPB angle constraints}
\label{subsubsec:RBP_and_RPB}

The RBP and RPB constraints are specified as shown below. The user may identify the reference body using either ``cb'' for the central body or an integer encoding a body's index in that Journey's Universe file. The lower and upper bounds on the constraint are specified in degrees.

The RBP and RPB constraints may only be specified for ephemeris-pegged arrival events that encode a $v_\infty$ vector, \textit{i.e.} intercept/flyby, chemical rendezvous, and orbit insertion. Note that because RBP and RPB only work with ephemeris-pegged boundary event, we can make the assumption that the ``body'' is the body associated with the boundary event and so we only need to encode the ``reference'' in the constraint description as shown below.

\begin{verbatim}
BEGIN_BOUNDARY_CONSTRAINT_BLOCK
p0_arrival_RBP_cb_0.0_15.0
p3_arrival_RPB_5_8.3_17.2
END_BOUNDARY_CONSTRAINT_BLOCK
\end{verbatim}


\subsubsection{RRP and RPR angle constraints}
\label{subsubsec:RRP_and_RP}

The RRP and RPR constraints are specified as shown below. The syntax for RRP and RPR requires that the user specify two reference bodies. The user may identify the reference body using either ``cb'' for the central body or an integer encoding a body's index in that Journey's Universe file. The lower and upper bounds on the constraint are specified in degrees.

The RRP and RPR constraints may be applied to any boundary event.

\begin{verbatim}
BEGIN_BOUNDARY_CONSTRAINT_BLOCK
p0_arrival_RRP_cb_3_0.0_15.0
p3_departure_RPR_5_7_8.3_17.2
END_BOUNDARY_CONSTRAINT_BLOCK
\end{verbatim}

\subsection{Angular Momentum Reference Angle Constraint}
\label{subsec:angular_momentum_reference_angle}

The user may choose to constrain the angle between the spacecraft's ICRF angular momentum vector relative to the central body and the vector to some reference body. For example, the user may need to constrain that the final orbit of the spacecraft about some body be in the terminator plane, \textit{i.e.} the angle between the vector to the sun and the angular momentum vector is zero.

The user may supply this constraint in degrees or radians.

\begin{verbatim}
BEGIN_BOUNDARY_CONSTRAINT_BLOCK
p0_arrival_angularMomentumReferenceAngle_cb_0.0deg_15.0deg
END_BOUNDARY_CONSTRAINT_BLOCK
\end{verbatim}


\subsection{Departure or Arrival Maneuver Thruster Set}
\label{subsec:boundary_constrainThrusterSet}

The user may direct a particular chemical departure or arrival maneuver to be either ``monoprop'' or ``biprop.'' EMTG will then use the $I_{sp}$ specified for monoprop or biprop in the spacecraft file.

This class of constraint is applied to departure type \texttt{0: launch or direct insertion} (when a launch vehicle is not used) and arrival types \texttt{0: insertion into parking orbit (use chemical Isp)} and \texttt{1: rendezvous (with chemical maneuver)}, as well as powered patched-conic flybys. All other boundary types will ignore this class of constraint.

\begin{verbatim}
BEGIN_BOUNDARY_CONSTRAINT_BLOCK
p0_arrival_monoprop
p3_departure_biprop
END_BOUNDARY_CONSTRAINT_BLOCK
\end{verbatim}

\subsection{Longitude}
\label{subsec:BCF_longitude}

The user may constrain longitude in ICRF, J2000BCF or TrueOfDateBCF. Longitude is computed as,

\begin{equation}
	\label{eq:longitude}
	\lambda = \atan2\left (y, x \right)
\end{equation}

\noindent and is given in the range $\left[-180.0, 180.0 \right]$. The user provides lower and upper bounds on longitude, in degrees.


\begin{verbatim}
BEGIN_BOUNDARY_CONSTRAINT_BLOCK
p0_departure_longitude_84.0_85.0
END_BOUNDARY_CONSTRAINT_BLOCK
\end{verbatim}

\subsection{BCF Latitude}
\label{subsec:BCF_latitude}

The user may constrain latitude in the J2000 \ac{BCF} frame as defined in the current journey's Universe file. This is also known as geocentric latitude as opposed to geodetic latitude. Latitude is computed as,

\begin{equation}
	\label{eq:latitude}
	\begin{aligned}
		\theta = \atan2\left (z_{J2000BCF}, r_{xy_{J2000BCF}} \right)\\
		r_{xy_{J2000BCF}} = \sqrt{\left(x_{J2000BCF}^2 + y_{J2000BCF}^2\right)}
	\end{aligned}
\end{equation}

\noindent and is given in the range $\left[-180.0, 180.0 \right]$. The user provides lower and upper bounds on longitude, in degrees.

Note that this constraint is computed in the \textit{universe's central body's} J2000 \ac{BCF} frame. This is true regardless of boundary class.

\begin{verbatim}
BEGIN_BOUNDARY_CONSTRAINT_BLOCK
p0_departure_BCFlatitude_-29.0_-28.0
END_BOUNDARY_CONSTRAINT_BLOCK
\end{verbatim}

\subsection{(Geo)Detic Latitude}
\label{subsec:detic_latitude}

The user may constrain (geo)detic latitude. The user may specify this constraint in J2000BCF and TrueOfDateBCI frames. The user provides lower and upper bounds on latitude, in degrees.
Note that this constraint has a singularity at the pole.

\begin{verbatim}
BEGIN_BOUNDARY_CONSTRAINT_BLOCK
p0_arrival_DeticLatitude_25.0_26.0_trueofdatebcf
END_BOUNDARY_CONSTRAINT_BLOCK
\end{verbatim}

\subsection{(Geo)Detic Altitude}
\label{subsec:detic_altitude}

The user may constrain (geo)detic altitude. The user may specify this constraint in J2000BCF and TrueOfDateBCI frames. The user provides lower and upper bounds on altitude, in km.
Note that (derivative of) this constraint has a singularity at the pole.

\begin{verbatim}
BEGIN_BOUNDARY_CONSTRAINT_BLOCK
p0_arrival_DeticAltitude_206.0_207.0_trueofdatebcf
END_BOUNDARY_CONSTRAINT_BLOCK
\end{verbatim}

\subsection{(Geo)Detic Target Body Elevation}
\label{subsec:target_detic_elevation}

The user may constrain the (geo)detic elevation of a target body with respect to the surface normal vector of a spheroid with flattening factor $f$. The target body is specified using an integer encoding a body's index in that Journey's Universe file. The user may specify this constraint in J2000BCF and TrueOfDateBCI frames. The user provides lower and upper bounds on detic elevation, in degrees.

%\noindent Note that (derivative of) this constraint has a singularity at the pole.

\begin{verbatim}
BEGIN_BOUNDARY_CONSTRAINT_BLOCK
p0_arrival_TargetDeticElevation_10_70.0_90.0_trueofdatebcf
END_BOUNDARY_CONSTRAINT_BLOCK
\end{verbatim}

\subsection{(Geo)Detic Elevation from Ground Station}
\label{subsec:detic_elevation_from_groundstation}

The user may constrain the (geo)detic elevation of the spacecraft with respect to a ground station. The location of the ground station is defined using latitude/longitude/altitude in a true-of-date, body-fixed reference frame relative to the body on which the ground station is located.

The syntax of the constraint following the \texttt{DeticElevationFromGroundStation} text block is:

\begin{enumerate}
	\item ID for body on which the ground station is located. Either \texttt{CB} for central body or a number for the body ID.
	\item Latitude of ground station.
	\item Longitude of ground station.
	\item Altitude of ground station.
	\item Lower bound on elevation angle.
	\item Upper bound on elevation angle.
\end{enumerate}

\begin{verbatim}
BEGIN_BOUNDARY_CONSTRAINT_BLOCK
p0_arrival_DeticElevationFromGroundStation_CB_28.455deg_-52.246deg_0.0km_60.0deg_60.0deg
END_BOUNDARY_CONSTRAINT_BLOCK
\end{verbatim}

\subsection{Velocity Declination}
\label{subsec:velocity_declination}

The user may constrain velocity declination, given by:

\begin{equation}
	\delta _v = \atan2 \left(v_z, v_{xy} \right)
\end{equation}

\noindent where $v_{xy}$ is the projection of the velocity vector onto the x-y plane. The user may specify this constraint in ICRF, J2000BCI, J2000BCF, TrueOfDateBCI, and TrueOfDateBCF frames.

\begin{verbatim}
BEGIN_BOUNDARY_CONSTRAINT_BLOCK
p0_departure_VelocityDeclination_-20.0_5.0_ICRF
END_BOUNDARY_CONSTRAINT_BLOCK
\end{verbatim}

\subsection{Velocity Declination with Respect to Any Body}
\label{subsec:velocity_declination_wrt_any_body}

The velocity declination constraint described in Section~\ref{subsec:velocity_declination} can only be used to constrain the velocity with respect to the central body of the journey. The constraint described here is the same mathematical construct, but can be referenced to any body defined in the relevant journey's universe.

\begin{verbatim}
BEGIN_BOUNDARY_CONSTRAINT_BLOCK
p<phase number>_<boundary><before or after boundary>_VelocityDeclinationAnyBody_
<Body ID>_<lower bound><lower bound units>_<upper bound><upper bound units>_<Frame>
END_BOUNDARY_CONSTRAINT_BLOCK
\end{verbatim}

where

\begin{itemize}
	\item \texttt{<phase number>} is the integer number of the phase on which the constraint is applied.
	\item \texttt{<boundary>} is \texttt{departure} or \texttt{arrival}.
	\item \texttt{<before or after boundary>} is \texttt{-} or \texttt{+}. A \texttt{-} indicates that the constraint is applied infinitesimally before the boundary. A \texttt{+} indicates that the constraint is applied infinitesimally after the boundary. See Section~\ref{sec:states_and_derivatives_before_and_after_boundaries} for a full description of what \texttt{-} or \texttt{+} actually means for a given boundary type/class.
	\item \texttt{<body ID>} is the ID for the body with respect to which the velocity declination is calculated. The velocity vector of the spacecraft is calculated with respect to this body, and the reference frame with respect to which the declination angle of the velocity vector is calculated is defined by this body. May be either \texttt{cb} to indicate the central body of the journey or any numerical body ID defined in the journey's universe file body list.
	\item \texttt{<lower bound>} The numerical lower bound for the constraint.
	\item \texttt{<lower bound units>} The units of the lower bound. May be \texttt{deg} or \texttt{rad}.
	\item \texttt{<upper bound>} The numerical upper bound for the constraint.
	\item \texttt{<upper bound units>} The units of the upper bound. May be \texttt{deg} or \texttt{rad}.
	\item \texttt{<Frame>} The reference frame with respect to which the declination angle is calculated, defined by \texttt{body ID}. Valid choices are \texttt{ICRF}, \texttt{J2000BCI}, \texttt{J2000BCF}, \texttt{TrueOfDateBCI}, and \texttt{TrueOfDateBCF}.
\end{itemize}

Syntax examples:

\begin{verbatim}
BEGIN_BOUNDARY_CONSTRAINT_BLOCK
p0_departure-_VelocityDeclinationAnyBody_CB_9_-1.0rad_2rad_ICRF
p0_arrival+_VelocityDeclinationAnyBody_3_10deg_50deg_TrueOfDateBCF
END_BOUNDARY_CONSTRAINT_BLOCK
\end{verbatim}

\subsection{Vertical Flight Path Angle (VFPA)}
\label{subsec:velocity_vertical_flight_path_angle}
The user may constrain vertical flight path angle, given by:

\begin{equation}
\psi = \arccos \left(\frac{\mathbf{r} \cdot \mathbf{v} }{rv}\right)
\end{equation}

\noindent where the $CF$ denotes the constraint frame. The user may specify this constraint in ICRF, J2000BCI, J2000BCF, TrueOfDateBCI, and TrueOfDateBCF frames.

\begin{verbatim}
BEGIN_BOUNDARY_CONSTRAINT_BLOCK
p0_arrival_VFPA_97.0_98.0_J2000BCF
END_BOUNDARY_CONSTRAINT_BLOCK
\end{verbatim}

\subsection{Velocity magnitude}
\label{subsec:velocity_magnitude}
The user may constrain velocity magnitude in km/s. The user may specify this constraint in ICRF, J2000BCI, J2000BCF, TrueOfDateBCI, and TrueOfDateBCF frames.

\begin{verbatim}
BEGIN_BOUNDARY_CONSTRAINT_BLOCK
p0_arrival_VelocityMagnitude_11.9_12.1_J2000BCF
END_BOUNDARY_CONSTRAINT_BLOCK
\end{verbatim}

\subsection{Velocity Spherical Azimuth}
\label{subsec:velocity_sphericalazimuth}
The user may constrain spherical azimuth angle of inertial velocity vector in degrees. The spherical azimuth angle is given as
\begin{equation}
	\label{eq:velocity_Az}
	\begin{aligned}
		&Az = \arctan{\frac{v_{east}}{v_{north}}}
	\end{aligned}
\end{equation}
,where $v_{east}$ and $v_{north}$ are inertial velocity elements at the spherical local coordinate as
\begin{equation}
	\label{eq:velocity_local}
	\begin{aligned}
		\begin{bmatrix}
		v_{up}\\
		v_{east}\\
		v_{north}
		\end{bmatrix} =
		\begin{bmatrix}
		\cos{\phi} & 0 & \sin{\phi}\\
		0 & 1 & 0\\
		-\sin{\phi} & 0 & \cos{\phi}
		\end{bmatrix}
		\begin{bmatrix}
		\cos{\lambda} & \sin{\lambda} & 0\\
		-\sin{\lambda}& \cos{\lambda} & 0\\
		0 & 0 & 1
		\end{bmatrix}
		\begin{bmatrix}
		v_{r}\\
		v_{r}\\
		v_{r}
		\end{bmatrix}
	\end{aligned}
\end{equation}
Here, $\phi$ is a (geo)centric latitude and $\lambda$ is a longitude. The user may specify this constraint in J2000BCF and TrueOfDateBCF frames.

\begin{verbatim}
BEGIN_BOUNDARY_CONSTRAINT_BLOCK
p0_arrival_VelocitySphericalAzimuth_50.1_50.2_trueofdatebcf
END_BOUNDARY_CONSTRAINT_BLOCK
\end{verbatim}

\subsection{Relative Velocity magnitude}
\label{subsec:relativevelocity_magnitude}
The user may constrain magnitude of velocity vector relative to central body's ground (or atmosphere) in km/s. The relative velocity vector and the magnitude are given by
\begin{equation}
	\label{eq:rel_velocity}
	\begin{aligned}
		\vec{v_r} = \vec{v} - \vec{\omega} \times \vec{r} = \vec{v} -
		\begin{bmatrix}
		0 & -\omega_3 & \omega_2 \\
		\omega_3 & 0 & -\omega_1 \\
		-\omega_2 & \omega_1 & 0
		\end{bmatrix}
		\vec{r}
	\end{aligned}
\end{equation}
\begin{equation}
	\label{eq:rel_velocity_magnitude}
	\begin{aligned}
		&v_r = \left|\vec{v_r}\right| = \sqrt{v_{rx}^{2}+v_{ry}^{2}+v_{rz}^{2}}
	\end{aligned}
\end{equation}
The user may specify this constraint in J2000BCF and TrueOfDateBCF frames.

\begin{verbatim}
BEGIN_BOUNDARY_CONSTRAINT_BLOCK
p0_arrival_RelativeVMagnitude_11.9_12.1_J2000BCF
END_BOUNDARY_CONSTRAINT_BLOCK
\end{verbatim}

\subsection{Relative Velocity Azimuth}
\label{subsec:relativevelocity_azimuth}
The user may constrain azimuth angle of velocity vector relative to central body's ground (or atmosphere) in degrees. The azimuth angle is given as
\begin{equation}
	\label{eq:rel_velocity_Az}
	\begin{aligned}
		&Az = \arctan{\frac{v_{r_{east}}}{v_{r_{north}}}}
	\end{aligned}
\end{equation}
,where $v_{r_{east}}$ and $v_{r_{north}}$ are relative velocity elements at the horizontal local coordinate as
\begin{equation}
	\label{eq:rel_velocity_local}
	\begin{aligned}
		\begin{bmatrix}
		v_{r_{up}}\\
		v_{r_{east}}\\
		v_{r_{north}}
		\end{bmatrix} =
		\begin{bmatrix}
		\cos{\phi} & 0 & \sin{\phi}\\
		0 & 1 & 0\\
		-\sin{\phi} & 0 & \cos{\phi}
		\end{bmatrix}
		\begin{bmatrix}
		\cos{\lambda} & \sin{\lambda} & 0\\
		-\sin{\lambda}& \cos{\lambda} & 0\\
		0 & 0 & 1
		\end{bmatrix}
		\begin{bmatrix}
		v_{rx}\\
		v_{ry}\\
		v_{rz}
		\end{bmatrix}
	\end{aligned}
\end{equation}
Here, $\phi$ is a (geo)detic latitude and $\lambda$ is a longitude. The user may specify this constraint in J2000BCF and TrueOfDateBCF frames.

\begin{verbatim}
BEGIN_BOUNDARY_CONSTRAINT_BLOCK
p0_arrival_RelativeVAzimuth_50.1_50.2_trueofdatebcf
END_BOUNDARY_CONSTRAINT_BLOCK
\end{verbatim}

\subsection{Relative Velocity Horizontal Flight Path Angle(HFPA)}
\label{subsec:relativevelocity_hfpa}
The user may constrain horizontal flight path angle of velocity vector relative to central body's ground (or atmosphere) in degrees. The HFPA is given as
\begin{equation}
	\label{eq:rel_velocity_HFPA}
	\begin{aligned}
		&HFPA = \arctan{\frac{v_{r_{up}}}{\sqrt{v_{r_{north}}^2 + v_{r_{east}}^2}}}
	\end{aligned}
\end{equation}
,where $v_{r_{east}}$ and $v_{r_{north}}$ are relative velocity elements at the horizontal local coordinate as
\begin{equation}
	\label{eq:rel_velocity_local}
	\begin{aligned}
		\begin{bmatrix}
		v_{r_{up}}\\
		v_{r_{east}}\\
		v_{r_{north}}
		\end{bmatrix} =
		\begin{bmatrix}
		\cos{\phi} & 0 & \sin{\phi}\\
		0 & 1 & 0\\
		-\sin{\phi} & 0 & \cos{\phi}
		\end{bmatrix}
		\begin{bmatrix}
		\cos{\lambda} & \sin{\lambda} & 0\\
		-\sin{\lambda}& \cos{\lambda} & 0\\
		0 & 0 & 1
		\end{bmatrix}
		\begin{bmatrix}
		v_{rx}\\
		v_{ry}\\
		v_{rz}
		\end{bmatrix}
	\end{aligned}
\end{equation}
Here, $\phi$ is a (geo)detic latitude and $\lambda$ is a longitude. The user may specify this constraint in J2000BCF and TrueOfDateBCF frames.

\begin{verbatim}
BEGIN_BOUNDARY_CONSTRAINT_BLOCK
p0_arrival_RelativeVHFPA_-14.1_-14.0_trueofdatebcf
END_BOUNDARY_CONSTRAINT_BLOCK
\end{verbatim}

\subsection{State in Two-Body Rotating Frame}
\label{subsec:state_in_two_body_rotating_frame}
The user may constrain the state of the spacecraft relative to a two-body rotating reference frame. This frame is analogous to, e.g., GMAT's ``ObjectReferenced'' frame type. The user sets two bodies in the journey's universe to define the axes of the frame. The $x$ axis points from the first body to the second body, the $z$ axis points in the direction of the angular momentum vector of the second body's motion about the first body, and the $y$ axis completes the right-handed set. Notes on implementation:

\begin{itemize}
	\item The user may constrain independently any element of the position/velocity state of the spacecraft in this frame.
	\item The constraint may be applied at arrival or departure of a phase. Additionally, the constraint may be applied infinitesimally before or after the phase boundary time in order to take into account (or not) events that happen instantaneously at a boundary (e.g., an impulsive maneuver).
	\item Currently, the constrained state can only be defined relative to the second body used to define the frame.
	\item Position constraints are expressed in the rotating frame.
	\item Velocity constraints are calculated by differentiating in the rotating frame \emph{and} expressing the velocity in the rotating frame.
\end{itemize} 

Syntax (all one line):

\begin{verbatim}
BEGIN_BOUNDARY_CONSTRAINT_BLOCK
p<phase number>_<boundary><before or after boundary>_StateInTwoBodyRotatingFrame
_<state element>_<body ID for B1>_<body ID for B2>_<origin of frame>
_<lower bound><lower bound units>_<upper bound><upper bound units>
END_BOUNDARY_CONSTRAINT_BLOCK
\end{verbatim}

where

\begin{itemize}
	\item \texttt{<phase number>} is the integer number of the phase on which the constraint is applied.
	\item \texttt{<boundary>} is \texttt{departure} or \texttt{arrival}.
	\item \texttt{<before or after boundary>} is \texttt{-} or \texttt{+}. A \texttt{-} indicates that the constraint is applied infinitesimally before the boundary. A \texttt{+} indicates that the constraint is applied infinitesimally after the boundary. See Section~\ref{sec:states_and_derivatives_before_and_after_boundaries} for a full description of what \texttt{-} or \texttt{+} actually means for a given boundary type/class.
	\item \texttt{<state element>} is the state to be constraint. Valid values are: \texttt{x}, \texttt{y}, \texttt{z}, \texttt{vx}, \texttt{vy}, \texttt{vz}.
	\item \texttt{<body ID for B1>} is the ID for the first body that defines the frame. May be either \texttt{cb} to indicate the central body of the journey or any numerical body ID defined in the journey's universe file body list. Cannot be the same as \texttt{<body ID for B2>}.
	\item \texttt{<body ID for B2>} is the ID for the second body that defines the frame. May be either \texttt{cb} to indicate the central body of the journey or any numerical body ID defined in the journey's universe file body list. \texttt{<body ID for B1>}.
	\item \texttt{<origin of frame>} Defines the origin of the rotating frame. Currently, the only valid value is \texttt{body2}.
	\item \texttt{<lower bound>} The numerical lower bound for the constraint.
	\item \texttt{<lower bound units>} The units of the lower bound. If \texttt{<state element>} is \texttt{x}, \texttt{y}, or \texttt{z}, then valid values are \texttt{km}, \texttt{AU}, or \texttt{LU}. If \texttt{<state element>} is \texttt{vx}, \texttt{vy}, or \texttt{vz}, then valid values are \texttt{km/s}.
	\item \texttt{<upper bound>} The numerical upper bound for the constraint.
	\item \texttt{<upper bound units>} The units of the upper bound. If \texttt{<state element>} is \texttt{x}, \texttt{y}, or \texttt{z}, then valid values are \texttt{km}, \texttt{AU}, or \texttt{LU}. If \texttt{<state element>} is \texttt{vx}, \texttt{vy}, or \texttt{vz}, then valid values are \texttt{km/s}.
\end{itemize}

Syntax examples:

\begin{verbatim}
BEGIN_BOUNDARY_CONSTRAINT_BLOCK
p0_departure-_StateInTwoBodyRotatingFrame_VY_CB_9_Body2_-1km/s_2km/s
p0_arrival+_StateInTwoBodyRotatingFrame_X_7_9_Body2_1000km_2000km
END_BOUNDARY_CONSTRAINT_BLOCK
\end{verbatim}

Additional documentation is available in the directory \texttt{docs/constraint\_in\_two\_body\_rotating\_frame}. The PowerPoint presentation \texttt{EMTG-182\_Constraint\_In\_Rotating\_Frame.pptx} describes the basic architectural design of the constraint. \texttt{constraint\_in\_two\_body\_rotating\_frame.pdf} gives the full mathematical specification.


\section{MGAnDSMs Maneuver Constraints}
\label{sec:MGAnDSMs_Maneuver_Constraints}

The \ac{MGAnDSMs} transcription allows the optimizer to place up to $n$ deep-space maneuvers in each phase of a multiple gravity-assist mission. Nominally, the numerical optimizer will place the maneuvers in locally optimal locations based on the \ac{NLP} necessary conditions for optimality. There are, however, many instances where the mission designer will want to constrain the parameters of a particular maneuver for practical reasons including, but not limited to:

\begin{enumerate}
	\item Navigation restrictions: no maneuver pre/post another mission critical event
	\item Engine type: restrict a certain maneuver to be executed with a particular propulsion system (bipropellant or monopropellant thruster sets)
	\item Magnitude: constrain the size of a particular maneuver due to hardware limitations or practical navigation concerns
\end{enumerate}

Individual maneuver constraints are specified in the .emtgopt file. Each journey has a maneuver constraint block:

\begin{verbatim}
BEGIN_MANEUVER_CONSTRAINT_BLOCK
constraint_1
constraint_2
constraint_3
.
.
.
constraint_n
END_MANEUVER_CONSTRAINT_BLOCK
\end{verbatim}

\noindent The basic maneuver constraint syntax identifies the journey and phase that the maneuver is in, as well as its index within the phase (indexed from zero).

\noindent Instructions on how to specify different types of maneuver constraints are provided in the following sections along with examples.

\subsection{Maneuver Epoch}
\label{subsec:MGAnDSMs_constraintEpoch}

Maneuver positions can be specified via their epoch in a variety of ways: an absolute epoch, epoch offset with respect to another mission event (e.g. another maneuver's epoch), or an epoch offset w.r.t. a right or left phase boundary.

\subsubsection{Absolute Epoch}

A maneuver may be fixed in time by specifying either a \ac{MJD} or \ac{JD}.

\begin{verbatim}
BEGIN_MANEUVER_CONSTRAINT_BLOCK
p0b0_epoch_abs_51544.0_51544.5
END_MANEUVER_CONSTRAINT_BLOCK
\end{verbatim}

\subsubsection{Epoch Relative to Phase Boundary}

A maneuver may be defined relative to the left or right boundary of the phase, in units of days.

\begin{verbatim}
BEGIN_MANEUVER_CONSTRAINT_BLOCK
p0b0_epoch_lboundary_14.0_10000.0
p0b0_epoch_rboundary_60.0_10000.0
END_MANEUVER_CONSTRAINT_BLOCK
\end{verbatim}

\subsubsection{Epoch Relative to Previous or Next Event}

A maneuver may be defined relative to the previous or next event in the phase (\textit{i.e.} another maneuver or a boundary), in units of days.

\begin{verbatim}
BEGIN_MANEUVER_CONSTRAINT_BLOCK
p0b0_epoch_next_14.0_10000.0
p0b0_epoch_previous_60.0_10000.0
END_MANEUVER_CONSTRAINT_BLOCK
\end{verbatim}


\subsection{Maneuver Magnitude}
\label{subsec:MGAnDSMs_constraintMagnitude}

The user may constrain the magnitude of any MGAnDSMs maneuver, in units of km/s.

\begin{verbatim}
BEGIN_MANEUVER_CONSTRAINT_BLOCK
p0b0_magnitude_0.0_0.01
END_MANEUVER_CONSTRAINT_BLOCK
\end{verbatim}

\subsection{Maneuver Thruster Set}
\label{subsec:MGAnDSMs_constrainThrusterSet}

The user may direct a particular MGAnDSMs maneuver to be either ``monoprop'' or ``biprop.'' EMTG will then use the $I_{sp}$ specified for monoprop or biprop in the spacecraft file.

\begin{verbatim}
BEGIN_MANEUVER_CONSTRAINT_BLOCK
p0b0_monoprop
END_MANEUVER_CONSTRAINT_BLOCK
\end{verbatim}


\section{Parallel Shooting Maneuver Constraints}
\label{subsec:PS_maneuver_constraints}

The parallel shooting transcriptions, \ac{PSFB} and \ac{PSBI} transcriptions decompose a low-thrust phase into \textit{n} subphases, or `steps.'' Each individual step may have up to \textit{m} maneuvers, or ``substeps,'' each of which has its own \ac{NLP} control parameters. As in \ac{MGAnDSMs}, constraints may be placed in the maneuver constraint block. \ac{PSFB} and \ac{PSBI} are designed such that constraints are placed at the step level, not at the substep level.

\subsection{Parallel Shooting Duty Cycle}
\label{subsec:PS_duty_cycle}

The user may directly constrain the duty cycle in each parallel shooting step as a floating point variable in [0.0, 1.0]. If EMTG is set to ``averaged'' duty cycle mode, this duty cycle will be applied as a multiplier to both thrust and mass flow rate. If EMTG is set to ``high-fidelity'' duty cycle mode, then the spacecraft will thrust for that percentage of the step length and then insert a forced coast until the end of the step.

\begin{verbatim}
BEGIN_MANEUVER_CONSTRAINT_BLOCK
p0b0_dutycycle_0.9
END_MANEUVER_CONSTRAINT_BLOCK
\end{verbatim}

\subsection{Parallel Shooting body-probe-thrust angle}
\label{subsec:PS_BPT}

The user may constrain the \ac{BPT} angle for any \ac{PSFB} or \ac{PSBI} phase of the CAESAR trajectory. This constraint applies to an entire phase rather than a particular step of that phase. For convergence reasons, the constraint is on the cosine of the BPT angle rather than the BPT angle itself.

The upper and lower bounds on $\cos \left(\theta_{BPT}\right)$ are expressed as functions of $r$, the distance between the spacecraft and a reference body of the user's choice. This formulation allows for BPT angle constraints that become stricter as the spacecraft flies closer to the reference body and less strict as the spacecraft flies away from the reference body. The upper and lower bounds are fit using the \ac{4PL} curve as per Equations \ref{eq:BPT_angle_bounds}.

\begin{equation}
	\label{eq:BPT_angle_bounds}
	\begin{aligned}
		L\left(r\right) \leq \cos \left(\theta_{BPT}\right) \leq U\left(r\right)\\
		U\left(r\right) = \alpha_{Ud} + \frac{\alpha_{Ua} - \alpha_{Ud}}{1.0 + \left(\frac{r}{\alpha_{Uc}}\right)^{\alpha_{Ub}}}\\
		L\left(r\right) = \alpha_{Ld} + \frac{\alpha_{La} - \alpha_{Ld}}{1.0 + \left(\frac{r}{\alpha_{Lc}}\right)^{\alpha_{Lb}}}
	\end{aligned}
\end{equation}

When specifying the constraint, the user must provide four coefficients each for the \ac{4PL} curves associated with the maximum and minimum values for $\cos \left(\theta_{BPT}\right)$, as well as the identity of the reference body. The reference body may be encoded as ``cb'' for the central body, or with an integer code describing the reference body's position in the Universe file. Note that the blow code block needs to be one line in the .emtgopt file.

Alternatively, the user may specify the constraint with a constant lower and upper bound. The bounds are given in degrees.

\begin{verbatim}
BEGIN_MANEUVER_CONSTRAINT_BLOCK
p0_bpt_cb_0.1879254,24.6087500,0.9480499,0.8878154...
           _1.6239600,14.9903500,0.8006161,-0.8878154
p0_bpt_cb_27.4_152.6
END_MANEUVER_CONSTRAINT_BLOCK
\end{verbatim}


\section{Phase Distance Constraint}
\label{sec:PhaseDistanceConstraint}

The user may constrain the distance between the spacecraft and any other body in the universe file at each time at which a maneuver occurs. The body may be expressed either as ``cb'' for the central body, or with an integer code describing the reference body's position in the Universe file. The lower and upper bounds on the distance constraint may be specified in LU, AU, or km.

At the moment, this constraint works in \ac{MGALT}, \ac{FBLT}, \ac{PSBI}, \ac{PSFB}, and \ac{MGAnDSMs}. Note that, for \ac{MGAnDSMs} especially (because its maneuvers are potentially spaced so far apart), this constraint does not necessarily mean that the spacecraft never violates this constraint. It means that no point at which the spacecraft performs a maneuver violates this constraint.

Each Journey block in the .emtgopt file has its own phase distance constraint section.

\begin{verbatim}
BEGIN_PHASE_DISTANCE_CONSTRAINT_BLOCK
p0_4_0.9au_10.0au
END_PHASE_DISTANCE_CONSTRAINT_BLOCK
\end{verbatim}

\section{PyEMTG Interface}
\label{sec:PyemtgInterface}

Scripted constraints may be accessed via PyEMTG through the following class properties.

\subsection{Boundary Constraints}
\label{sec:PyemtgBoundaryConstraints}

Boundary constraint definitions are stored on a per-journey basis as a list in

\begin{verbatim}
JourneyOptions.BoundaryConstraintDefinitions
\end{verbatim}
 
Each element of the list is a string that contains the entirety of the boundary constraint text. In other words, the upper and lower bound, etc. of the constraint are not stored separately. If there are no boundary constraints for a given journey, then the list is empty.

Boundary constraint output is stored on a per-journey basis as a list in

\begin{verbatim}
Journey.BoundaryConstraintOutputs
\end{verbatim}

Each element of the list is a string that contains the entirety of the boundary constraint output text (i.e., the entirety of the line written to the .emtg output file for that constraint). The user may then employ string-parsing techniques to, e.g., extract the numerical value of the constraint. If there are no boundary constraints for a given journey, then the list is empty.

\subsection{Maneuver Constraints}
\label{sec:PyemtgManueverConstraints}

Maneuver constraint definitions are stored on a per-journey basis as a list in

\begin{verbatim}
JourneyOptions.ManeuverConstraintDefinitions
\end{verbatim}

Each element of the list is a string that contains the entirety of the maneuver constraint text. In other words, the upper and lower bound, etc. of the constraint are not stored separately. If there are no maneuver constraints for a given journey, then the list is empty.

\subsection{Phase Distance Constraints}
\label{sec:PyemtgPhaseDistanceConstraints}

Phase distance constraint definitions are stored on a per-journey basis as a list in

\begin{verbatim}
JourneyOptions.PhaseDistanceConstraintDefinitions
\end{verbatim}

Each element of the list is a string that contains the entirety of the phase distance constraint text. In other words, the upper and lower bound, etc. of the constraint are not stored separately. If there are no phase distance constraints for a given journey, then the list is empty.

%\bibliographystyle{AAS_publication}
%\bibliography{EMTGbib_Jacob_June_2014}



\end{document}
