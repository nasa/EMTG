%%%%%%%%%%%%%%%%%%%%%%%%
%\section{PEATSA}
%\label{sec:peatsa}
%%%%%%%%%%%%%%%%%%%%%%%%

\ac{PEATSA} is a set of Python scripts used to create and execute \ac{EMTG} trade studies. \ac{PEATSA} operates by taking an \ac{EMTG} options file (the ``base case''\footnote{nomenclature that may be unfamiliar to a non-\ac{PEATSA} user is written in quotes on first reference.}) and executing variations of some of those options. The user sets the different variations prior to exectution. For example, a user might vary the launch date, the launch vehicle, a gravity-assist sequence, and/or a maximum time of flight. One execution of all desired variations constitutes a ``\ac{PEATSA} iteration.'' At the end of each iteration, a user-defined objective function is evaluated for each \ac{EMTG} case, and this value is compared against the best objective function value achieved in any previous iteration for each \ac{EMTG} case. The \ac{EMTG} execution with the best value of the objective function for each case is saved, and information about the best execution for each case is written to a summary file.

\noindent For the next iteration, initial guesses for \ac{EMTG} cases are generated based on existing results from cases with similar option variations. For example, if the user is varying the launch date (i.e., the launch date is a trade parameter), then the user can allow a case to use an initial guess solution from another case (a ``neighbor'') as long as the launch dates of the two cases are within the vicinity of  \(\sim \)5 days of each other, and all other trade parameters are the same between the two cases. This process is called ``seeding'' or using one case to ``seed'' another. The idea behind seeding is that similar solutions are likely to exist between neighboring cases, so using a neighbor as a seed is likely to result in a feasible solution. In this way, a solution for a single case can turn into a ``family'' of solutions after a number of iterations as neighbor $i$ seeds neighbors $i+1$ and $i-1$ on iteration $j$, and then neighbor $i+1$ seeds neighbors $i$ and $i+2$ on iteration $j+1$. Note that in the preceding description, a neighbor that is used as a seed may also be seeded by a neighbor of its own. As a result, any new optimal solution can ``propagate'' to create a new solution family.

\noindent The remainder of this section decribes how to create, execute, and analyze the results of a \ac{PEATSA} run.