%%%%%%%%%%%%%%%%%%%%%%%%
%\subsection{CSPICE}
%\label{sec:cspice}
%%%%%%%%%%%%%%%%%%%%%%%%

\subsubsection{Purpose}

\ac{EMTG} depends on CSPICE for ephemeris-lookup utilities, and is known to work with CSPICE N0067. 

\subsubsection{Download Location}

The version of CSPICE that is known to be compatible with \ac{EMTG} can be found at \url{https://naif.jpl.nasa.gov/pub/naif/toolkit/C/}.

\subsubsection{Dependency Installation Instructions}

\begin{enumerate}
	\item Navigate to the Utilities directory by executing the following command: \\

	\texttt{cd /Utilities/}
	\item Download CSPICE N0067 using the following command:

	\texttt{curl -LO https://naif.jpl.nasa.gov/pub/naif/toolkit/C/PC\_Linux\_GCC\_64bit/packages\newline\indent /cspice.tar.Z}

	\item Extract the tarball and navigate into the newly created directory using the following commands:
	\begin{verbatim}
	tar -xzf cspice.tar.Z

	cd cspice
	\end{verbatim}   

	\item Build the final executable using the following command:
	\begin{verbatim}
	./makeall.csh
	\end{verbatim}
	\begin{itemize}
		\item If you encounter an error when running the \texttt{makeall.csh} script, you will have to build each component individually by navigating to each of the subdirectories within the \texttt{cspice/src/} directory and running the make script there. For example, the following commands will make the cspice core:
		\begin{verbatim}
		cd cspice/src/cspice

		./mkprodct.csh
		\end{verbatim}
	\end{itemize}
\end{enumerate}