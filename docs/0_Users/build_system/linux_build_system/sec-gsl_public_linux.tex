%%%%%%%%%%%%%%%%%%%%%%%%
%\subsubsection{\ac{GSL}}
%\label{sec:gsl}
%%%%%%%%%%%%%%%%%%%%%%%%

\subsubsection{Purpose}

\ac{EMTG} depends on \ac{GSL} for cubic-splining utilities. \ac{EMTG} is known to work with \ac{GSL} 2.7.0.  If you already installed \ac{GSL} in Section~\ref{sec:setting_up_dependencies_with_management_capabilities}, skip this section. 

\subsubsection{Download Location}

There are multiple ways to get \ac{GSL}, but the method supported by the \ac{EMTG} build system is to obtain the AMPL version, which has a CMake-based build system. (For this reason, CMake must be installed before \ac{GSL}.) The version of \ac{GSL} known to be compatible with \ac{EMTG} is available at \url{https://github.com/ampl/gsl/releases/tag/20211111}. 

\subsubsection{Dependency Installation Instructions}

\begin{enumerate}
	\item Navigate to the Utilities directory by executing the following command: \\

	\texttt{cd /Utilities/}
	\item Download AMPL \ac{GSL} 2.7.0 using the following command:\\

	\texttt{curl -LO https://github.com/ampl/gsl/archive/refs/tags/20211111.tar.gz}

	\item Extract the tarball using the following commands:
	\begin{verbatim}
	tar -xzf 20211111.tar.gz
	\end{verbatim}
	\item Rename the GSL directory and navigate into it using the following commands:
	\begin{verbatim}
	mv /Utilities/gsl-20211111/ /Utilities/gsl-2.7.0/	
	cd gsl-2.7.0
	\end{verbatim}

	\item Create a build directory and navigate into it using the following commands:
	\begin{verbatim}
	mkdir build
	cd build
	\end{verbatim}
	
	\item Build the final executable using the following commands: \\
	\textit{NOTE: \texttt{<number-of-cores-available>} is the integer number of cores to be used to perform the build.}
	
	\begin{verbatim}
	cmake .. -DNO_AMPL_BINDINGS=1
	make -j <number-of-cores-available>
	\end{verbatim}
	
\end{enumerate}