\documentclass[]{article}
\usepackage{nomencl}
\usepackage{hyperref}
\hypersetup{pdffitwindow=true,
	pdfpagemode=UseThumbs,
	breaklinks=true,
	colorlinks=true,
	linkcolor=black,
	citecolor=black,
	filecolor=black,
	urlcolor=black}

\usepackage{verbatim}
\usepackage[T1]{fontenc}
\usepackage{graphicx}%
\usepackage{amsmath}
\usepackage{amssymb}
\usepackage{amsthm}
\usepackage{subfigure}
\usepackage[makeroom]{cancel}
\usepackage{indentfirst}
\usepackage{color}
\usepackage{bm}
\usepackage{mathtools}

\usepackage{rotating}
\usepackage{comment}
\usepackage{here}
\usepackage{tabularx}
\usepackage{multirow}
\usepackage{setspace}
\usepackage{pdfpages}
\usepackage{float}
\usepackage[section]{placeins}

\usepackage{array}

% quotes
\newcommand{\quotes}[1]{``#1''}

% vectors and such
\newcommand{\vb}[1]{\bm{#1}} % bold
\newcommand{\vbd}[1]{\dot{\bm{#1}}} % dot
\newcommand{\vbdd}[1]{\ddot{\bm{#1}}} % double dot
\newcommand{\vbh}[1]{\hat{\bm{#1}}} % hat
\newcommand{\vbt}[1]{\tilde{\bm{#1}}} % tilde
\newcommand{\vbth}[1]{\hat{\tilde{\bm{#1}}}} % tilde hat
\newcommand{\ddt}[1]{\frac{\mathrm{d} #1}{\mathrm{d} t}} % time derivative
\newcommand{\pd}[2]{\frac{\partial #1}{\partial #2}} % partial derivative
\newcommand{\crossmat}[1]{\left\{ {#1} \right\}^{\times}} % crossmat
\newcommand{\xb}[0]{\vb{x}_b}
\newcommand{\xc}[0]{\vb{x}_c}
\newcommand{\diff}{\mathrm{d}}

\makenomenclature
\makeindex

%opening
\title{The Russell Transformation}
\author{Noble Hatten}

\begin{document}

\maketitle

\begin{abstract}
This document describes the ``Russell'' transformation, which is used to produce a fictitious independent variable for step-size control in $N$-body gravitational environments.

\end{abstract}

\tableofcontents

\printnomenclature

%%%%%%%%%%%%%%%%%%%%%%%%%%%%%%%%%%%%%%%%%%%%%%%%%%%%%%%%%%%%%%%%%%%%%%%%%%%%%%%
\section{The Russell Transformation}
\label{sec:russell_transformation}
%%%%%%%%%%%%%%%%%%%%%%%%%%%%%%%%%%%%%%%%%%%%%%%%%%%%%%%%%%%%%%%%%%%%%%%%%%%%%%%

The transformation takes the form

\begin{align}
	\diff t &= g \diff s,
\end{align}

\nomenclature{$t$}{Time}
\nomenclature{$s$}{Fictitious independent variable}
\nomenclature{$g$}{$\diff t / \diff s$}

\noindent where

\begin{align}
	g &= \prod_{i=1}^{N} \rho_i^{\alpha},
\end{align}

\nomenclature{$i$}{Index}
\nomenclature{$N$}{Number of gravitating bodies}
\nomenclature{$\alpha$}{User-tunable parameter; Sundman-like exponent}
\nomenclature{$\rho$}{Term in transformation}

\noindent where

\begin{align}
	\rho_i &= \frac{A^2 + A}{A + \left( \frac{A \left( 1 - C \right)}{C + A} \right)^{\frac{r_i}{B r_{H, i}}}} - A.
\end{align}

\nomenclature{$A$}{User-tunable parameter; scales how much time slows down at Hill-normalized distance from a body}
\nomenclature{$B$}{User-tunable parameter; defines boundary in space at which time starts to slow down as spacecraft approaches a body}
\nomenclature{$C$}{User-tunable parameter; defines values of $\rho$ when $r/r_H = B$}
\nomenclature{$r_i$}{Distance from spacecraft to body $i$}
\nomenclature{$r_H$}{Hill radius}

\noindent Then, the equations of motion are written as

\begin{align}
	\vb{f}_s &= \frac{\diff \vb{x}}{\diff s} \\
	&= \vb{f}_t \frac{\diff t}{\diff s} \\
	&= \vb{f}_t g
\end{align}

\nomenclature{$\vb{f}_t$}{$\frac{\diff \vb{x}}{\diff t}$}
\nomenclature{$\vb{f}_s$}{$\frac{\diff \vb{x}}{\diff s}$}
\nomenclature{$\vb{x}$}{State vector}

%%%%%%%%%%%%%%%%%%%%%%%%%%%%%%%%%%%%%%%%%%%%%%%%%%%%%%%%%%%%%%%%%%%%%%%%%%%%%%%
\section{Jacobian of the Russell Transformation}
\label{sec:russell_transformation_jacobian}
%%%%%%%%%%%%%%%%%%%%%%%%%%%%%%%%%%%%%%%%%%%%%%%%%%%%%%%%%%%%%%%%%%%%%%%%%%%%%%%

We require the Jacobians $\pd{\vb{f}_s}{\vb{x}}$ and $\pd{\vb{f}_s}{s}$.

\begin{align}
	\pd{\vb{f}_s}{\vb{x}} &= \pd{}{\vb{x}} \left[ \vb{f}_t g \right] \\
	&= \pd{\vb{f}_t}{\vb{x}} g + \vb{f}_t \pd{g}{\vb{x}} \\
	\pd{\vb{f}_s}{s} &= \pd{\vb{f}_s}{t} \frac{\diff t}{\diff s} \\
	&= \pd{}{t} \left[ \vb{f}_t g \right] g \\
	&= \pd{\vb{f}_t}{t} g^2 + \vb{f}_t \pd{g}{t} g
\end{align}

\noindent $\pd{\vb{f}_t}{\vb{x}}$ and $\pd{\vb{f}_t}{t}$ are the usual Jacobians for equations of motion when time is the independent variable, and $\vb{f}_t$ are the time equations of motion. Thus, the only terms that need to be derived are $\pd{g}{\vb{x}}$ and $\pd{g}{t}$.

%%%%%%%%%%%%%%%%%%%%%%%%%%%%%%%%%%%%%%%%%%%%%%%%%%%%%%%%%%%%%%%%%%%%%%%%%%%%%%%
\section{Jacobian of $\rho_i$}
\label{sec:rhoi_jacobian}
%%%%%%%%%%%%%%%%%%%%%%%%%%%%%%%%%%%%%%%%%%%%%%%%%%%%%%%%%%%%%%%%%%%%%%%%%%%%%%%

In $g$, the only non-constant terms are the $\rho_i$. Each $\rho_i$ is solely dependent on $r_i$. Using MATLAB symbolic math, we arrive at

\begin{align}
	\pd{\rho_i}{r_i} &= -\frac{A \gamma^{\frac{r_i}{B r_{H,i}}} \log \left( \gamma \right) \left(A + 1 \right)}{B r_{H, i} \left( A + \gamma^{\frac{r_i}{B r_{H,i}}} \right)^2} \\
	\gamma &\triangleq \frac{A \left( 1 - C \right)}{C + A},
\end{align}

\noindent where $\gamma$ is a convenience function. Then, $r_i$ is only directly dependent on $\vb{r}_i$:

\begin{align}
	\pd{r_i}{\vb{x}} &= \pd{r_i}{\vb{r}_i} \pd{\vb{r}_i}{\vb{x}} \\
	\pd{r_i}{\vb{r}_i} &= \frac{\vb{r}_i^T}{r_i}.
\end{align}

If we call $\vb{r}_i$ the vector from the spacecraft to body $i$ and $\vb{r}_{c, i}$ the vector from the central body of integration to body $i$, then

\begin{align}
	\vb{r}_i &= \vb{r}_{c, i} - \vb{r}.
\end{align}

\nomenclature{$\vb{r}$}{Postion vector of spacecraft}
\nomenclature{$\vb{r}_i$}{Position vector from spacecraft to body $i$}
\nomenclature{$\vb{r}_{c,i}$}{Position vector from central body to body $i$}

The derivatives are

\begin{align}
	\label{eq:dridx}
	\pd{\vb{r}_i}{\vb{x}} &= \pd{}{\vb{x}} \left( \vb{r}_{c,i} - \vb{r} \right).
\end{align}

\noindent In Eq.~\eqref{eq:dridx}, $\vb{r}_{c, i}$ depends only on time, and $\vb{r}$ depends only on the position state of the spacecraft. So,

\begin{align}
	\pd{\vb{r}_i}{t} &= \vb{v}_{c, i} \\
	\pd{\vb{r}_i}{\vb{r}} &= - \vb{I}
\end{align}

\noindent and all other derivatives are zero.

So,

\begin{align}
	\pd{\rho_i}{\vb{x}} &= \pd{\rho_i}{r_i} \pd{r_i}{\vb{x}} \pd{\vb{r}_i}{\vb{x}},
\end{align}

\noindent where we may assume that $t$ is part of the augmented state vector.

%%%%%%%%%%%%%%%%%%%%%%%%%%%%%%%%%%%%%%%%%%%%%%%%%%%%%%%%%%%%%%%%%%%%%%%%%%%%%%%
\section{Jacobian of $g$}
\label{sec:g_jacobian}
%%%%%%%%%%%%%%%%%%%%%%%%%%%%%%%%%%%%%%%%%%%%%%%%%%%%%%%%%%%%%%%%%%%%%%%%%%%%%%%

The derivative of $g$ with respect to each of the $\rho_i$ is

\begin{align}
	\pd{g}{\rho_i} &= \alpha \rho_i^{\alpha - 1} \prod_{\substack{j=1 \\ j \neq i}}^{N} \rho_j^\alpha.
\end{align}

\nomenclature{$j$}{Index}

Then, if we concatenate all $\rho_i$ into the vector $\vb{\rho}$, we finally get

\begin{align}
	\pd{g}{\vb{x}} &= \pd{g}{\vb{\rho}} \pd{\vb{\rho}}{\vb{x}},
\end{align}

\noindent where we may assume that $t$ is part of the augmented state vector.


%\bibliography{}
%\bibliographystyle{plain}

\end{document}