\documentclass[11pt]{article}

%% PACKAGES
\usepackage{graphicx}
\usepackage[printonlyused]{acronym}
\usepackage{float}
\usepackage[colorlinks=false]{hyperref}
\usepackage{tabularx}
\usepackage{caption}
\usepackage[margin=1.0in]{geometry}
\usepackage{tocloft}
\usepackage{listings}

\lstset{basicstyle=\small\ttfamily,columns=flexible,breaklines=true,xleftmargin=0.5in,keepspaces=true}


\makeatletter
\g@addto@macro\normalsize{%
  \setlength\abovedisplayskip{0.25pt}
  \setlength\belowdisplayskip{0.25pt}
  \setlength\abovedisplayshortskip{0.25pt}
  \setlength\belowdisplayshortskip{0.25pt}
}
\makeatother

\setlength{\parskip}{\baselineskip}

%% GRAPHICS PATH
\graphicspath{{../../shared_latex_inputs/images}{../../shared_latex_inputs/graphs}}

\newcommand{\acposs}[1]{%
	\expandafter\ifx\csname AC@#1\endcsname\AC@used
	\acs{#1}'s%
	\else
	\aclu{#1}'s (\acs{#1}'s)%
	\fi
}

\title{\Huge PEATSA Test Tutorial}

\newcommand{\listofknownissuesname}{\Large List of Known Issues}
\newlistof{knownissues}{mcf}{\listofknownissuesname}

\newcommand{\knownissue}[3]
{
	\refstepcounter{knownissues}
	\par\noindent\textbf{\hyperref[#2_b]{\theknownissues\quad #1}}\label{#2_h}
	\textbf{\hfill\pageref{#2_b}}
	#3
}

\newcommand{\knownissuelabel}[2]
{
	 \phantomsection
  	\hyperref[#2_h]{#1}\def\@currentlabel{\unexpanded{#1}}\label{#2_b}
}


\begin{document}

\begin{titlepage}
\maketitle
\thispagestyle{empty}
\begin{table}[H]
	\centering
	\begin{tabularx}{\textwidth}{|l|l|X|}
		\hline
		\textbf{Revision Date} & \textbf{Author} & \textbf{Description of Change} \\
		\hline
		\date{April 25, 2024} & Alec Mudek & Initial revision.\\ 
		\hline
	\end{tabularx}
\end{table}
\end{titlepage}

\newpage
\tableofcontents
\thispagestyle{empty}
\newpage

\listofknownissues
\thispagestyle{empty}

%\knownissue{An \acs{NLP} test is incorrectly listed as failed}{known_failing_test}
%\knownissue{Some tests fail if EMTG is not in the C:\textbackslash emtg\textbackslash{} directory}{tests_require_specific_dir}

\newpage
\clearpage
\setcounter{page}{1}



\section*{List of Acronyms}
\begin{acronym}
%To define the acronym and include it in the list of acronyms: \acro{acronym}{definition}
%To define the acronym and exclude it from the list of acronyms:  \acro{acronym}{definition}
%
%\ac{acronym} Expand and identify the acronym the first time; use only the acronym thereafter
%\acf{acronym} Use the full name of the acronym.
%\acs{acronym} Use the acronym, even before the first corresponding \ac command
%\acl{acronym}  Expand the acronym without using the acronym itself.
%
%

\acro{ACO}{Ant Colony Optimization}
\acro{AD}{Automatic Differentiation}
\acro{ADL}{Architecture Design Laboratory}
\acro{AES}{Advanced Exploration Systems}
\acro{AGA}{aerogravity assist}
\acro{ALARA}{As Low As Reasonably Achievable}
\acro{API}{application programming interface}
\acro{BB}{branch and bound}
\acro{BVP}{Boundary Value Problem}
\acro{CATO}{Computer Algorithm for Trajectory Optimization}
\acro{CL}{confidence level}
\acro{CONOPS}{concept of operations}
\acro{COV}{Calculus of Variations}
\acro{D/AV}{Descent/Ascent Vehicle}
\acro{DE}{Differential Evolution}
\acro{DLA}{Declination of Launch Asymptote}
\acro{DPTRAJ/ODP}{Double Precision Trajectory and Orbit Determination Program}
\acro{DSH}{Deep Space Habitat}
\acro{DSN}{Deep Space Network}
\acro{DSMPGA}{Dynamic-Size Multiple Population Genetic Algorithm}
\acro{EB}{Evolutionary Branching}
\acro{ECLSS}{environmental control and life support system}
\acro{ELV}{expendable launch vehicle}
\acro{EMME}{Earth to Mars, Mars to Earth}
\acro{EMMVE}{Earth to Mars, Mars to Venus to Earth}
\acro{EMTG}{Evolutionary Mission Trajectory Generator}
\acro{EVMME}{Earth to Venus to Mars, Mars to Earth}
\acro{EVMMVE}{Earth to Venus to Mars, Mars to Venus to Earth}
\acro{ERRV}{Earth Return Re-entry Vehicle}
\acro{FISO}{Future In-Space Operations}
\acro{FMT}{Fast Mars Transfer}
\acro{GASP}{Gravity Assist Space Pruning}
\acro{GCR}{galactic cosmic radiation}
\acro{GRASP}{Greedy Randomized Adaptive Search Procedure}
\acro{GSFC}{Goddard Space Flight Center}
\acro{GTOC}{Global Trajectory Optimization Competition}
\acro{GTOP}{Global Trajectory Optimization Problem}
\acro{HAT}{Human Architecture Team}
\acro{HGGA}{Hidden Genes Genetic Algorithm}
\acro{IMLEO}{Initial Mass in \acl{LEO}}
\acro{IPOPT}{Interior Point OPTimizer}
\acro{ISS}{International Space Station}
\acro{JHUAPL}{Johns Hopkins University Applied Physics Laboratory}
\acro{JSC}{Johnson Space Center}
\acro{KKT}{Karush-Kuhn-Tucker}
\acro{LEO}{Low Earth Orbit}
\acro{LRTS}{lazy race tree search}
\acro{MONTE}{Mission analysis, Operations, and Navigation Toolkit Environment}
\acro{MCTS}{Monte Carlo tree search}
\acro{MGA}{Multiple Gravity Assist}
\acro{MIRAGE}{Multiple Interferometric Ranging Analysis using GPS Ensemble}
\acro{MOGA}{Multi-Objective Genetic Algorithm}
\acro{MOSES}{Multiple Orbit Satellite Encounter Software}
\acro{MPI}{message passing interface}
\acro{MPLM}{Multi-Purpose Logistics Module}
\acro{MSFC}{Marshall Space Flight Center}
\acro{NELLS}{NASA Exhaustive Lambert Lattice Search}
\acro{NSGA}{Non-Dominated Sorting Genetic Algorithm}
\acro{NSGA-II}{Non-Dominated Sorting Genetic Algorithm II}
\acro{NHATS}{Near-Earth Object Human Space Flight Accessible Targets Study}
\acro{NTP}{Nuclear Thermal Propulsion}
\acro{OD}{orbit determination}
\acro{OOS}{On-Orbit Staging}
\acro{PCC}{Pork Chop Contour}
\acro{PEL}{permissible exposure limits}
\acro{PLATO}{PLAnetary Trajectory Optimization}
\acro{REID}{risk of exposure-induced death}
\acro{RTBP}{Restricted Three Body Problem}
\acro{SA}{Simulated Annealing}
\acro{SLS}{Space Launch System}
\acro{SNOPT}{Sparse Nonlinear OPTimizer}
\acro{SOI}{sphere of influence}
\acro{SPE}{solar particle events}
\acro{SQP}{sequential quadratic programming}
\acro{SRAG}{Space Radiation Analysis Group}
\acro{TEI}{Trans-Earth Injection}
\acro{TOF}{time of flight}
\acro{TPBVP}{Two Point Boundary Value Problem}
\acro{TMI}{Trans-Mars Injection}
\acro{VARITOP}{Variational calculus Trajectory Optimization Program}
\acro{VILM}{v-infinity leveraging maneuver}
\acro{MOI}{Mar Orbit Injection}
\acro{PCM}{Pressurized Cargo Module}
\acro{STS}{Space Transportation System}
\acro{EDS}{Earth Departure Stage}
\acro{NEO}{near-Earth asteroid}
\acro{IDC}{Integrated Design Center}
\acro{SEP}{solar-electric propulsion}
\acro{SRP}{solar radiation pressure}
\acro{NEP}{nuclear-electric propulsion}
\acro{REP}{radioisotope-electric propulsion}
\acro{DRM}{Design Reference Missions}

\acro{ASCII}{American Standard Code for Information Interchange}
\acro{AU}{Astronomical Unit}
\acro{BWG}{Beam Waveguides}
\acro{CCB}{Configuration Control Board}
\acro{CMO}{Configuration Management Office}
\acro{CODATA}{Committee on Data for Science and Technology}
\acro{DEEVE}{Dynamically Equivalent Equal Volume Ellipsoid}
\acro{DRA}{Design Reference Asteroid}
\acro{EME2000}{Earth Centered, Earth Mean Equator and Equinox of J2000 (Coordinate Frame)}
\acro{EOP}{Earth Orientation Parameters}
\acro{ET}{Ephemeris Time}
\acro{FDS}{Flight Dynamics System}
\acro{FTP}{File Transfer Protocol}
\acro{GSFC}{Goddard Space Flight Center}
\acro{PI}{Principal Investigator}
\acro{HEF}{High Efficiency}
\acro{IAG}{International Association of Geodesy}
\acro{IAU}{International Astronomical Union}
\acro{IERS}{International Earth Rotation and Reference Systems Service}
\acro{ICRF}{International Celestial Reference Frame}
\acro{ITRF}{International Terrestrial Reference System}
\acro{IOM}{Interoffice Memorandum}
\acro{JD}{Julian Date}
\acro{JPL}{Jet Propulsion Laboratory}
\acro{LM}{Lockheed Martin}
%\acro{LP150Q}{}
%\acros{LP100K}{}
\acro{MAVEN}{Mars Atmosphere and Volatile EvolutioN}
\acro{MJD}{Modified Julian Date}
\acro{MOID}{Minimum Orbit Intersection Distance}
\acro{MPC}{Minor Planet Center}
\acro{NASA}{National Aeronautics and Space Administration}
\acro{NDOSL}{\ac{NASA} Directory of Station Locations}
\acro{NEA}{near-Earth asteroid}
\acro{NEO}{near-Earth object}
\acro{NIO}{Nav IO}
\acro{OSIRIS-REx}{Origins Spectral Interpretation Resource Identification Security-Regolith Explorer}
\acro{PHA}{Potentially Hazardous Asteroid}
\acro{PHO}{Potentially Hazardous Object}
\acro{SBDB}{Small-Body Database}
\acro{SI}{International System of Units}
\acro{SPICE}{Spacecraft Planet Instrument Camera-matrix Events}
\acro{SPK}{SPICE Kernel}
\acro{SRC}{Sample Return Capsule}
\acro{SSD}{Solar System Dynamics}
\acro{STK}{Systems Tool Kit}
\acro{TAI}{International Atomic Time}
\acro{TBD}{To Be Determined}
\acro{TBR}{To Be Reviewed}
\acro{TCB}{Barycentric Coordinate Time}
\acro{TDB}{Temps Dynamiques Barycentrique, Barycentric Dynamical Time}
\acro{TDT}{Terrestrial Dynamical Time}
\acro{TT}{Terrestrial Time}
\acro{URL}{Uniform Resource Locator}
\acro{UT}{Universal Time}
\acro{UT1}{Universal Time Corrected for Polar Motion}
\acro{UTC}{Coordinated Universal Time}
\acro{USNO}{U. S. Naval Observatory}
\acro{YORP}{Yarkovsky-O'Keefe-Radzievskii-Paddack}

\acro{NLP}{nonlinear program}
\acro{MBH}{monotonic basin hopping}
\acro{MBH-C}{monotonic basin hopping with Cauchy hops}
\acro{FBS}{forward-backward shooting}
\acro{MGALT}{multiple gravity assist with low-thrust}
\acro{MGALTS}{multiple gravity assist with low-thrust using a Sundman transformation}
\acro{MGA-1DSM}{Multiple Gravity Assist with One Deep Space Maneuver}
\acro{MGAnDSMs}{multiple gravity assist with n deep-space maneuvers using shooting}
\acro{PSFB}{Parallel Shooting with Finite-Burn}
\acro{PSBI}{Parallel Shooting with Bounded Impulses}
\acro{FBLT}{finite-burn low-thrust}
\acro{ESA}{European Space Agency}
\acro{ACT}{Advanced Concepts Team}
\acro{IRAD}{independent research and development}
%\acro{Isp}[$\text{I}_{sp}$]{specific impulse}
\acro{GA}{genetic algorithm}
\acro{GALLOP}{ Gravity Assisted Low-thrust Local Optimization Program}
\acro{MALTO}{Mission Analysis Low-Thrust Optimization}
\acro{PaGMO}{Parallel Global Multiobjective Optimizer}
\acro{FRA}{feasible region analysis}
\acro{CP}{conditional penalty}
\acro{HOC}{hybrid optimal control}
\acro{HOCP}{hybrid optimal control problem}
\acro{PSO}{particle swarm optimization}
\acro{SEPTOP}{Solar Electric Propulsion Trajectory Optimization Program}
\acro{STOUR}{Satellite Tour Design Program}
\acro{STOUR-LTGA}{Satellite Tour Design Program - Low Thrust, Gravity Assist}
\acro{PaGMO}{Parallel Global Multiobjective Optimizer}
\acro{SDC}{static/dynamic control}
\acro{DDP}{Differential Dynamic Programming}
\acro{HDDP}{Hybrid Differential Dynamic Programming}
\acro{ACT}{Advanced Concepts Team}
\acro{GMAT}{General Mission Analysis Toolkit}
\acro{BOL}{beginning of life}
\acro{EOL}{end of life}
\acro{KSC}{Kennedy Space Center}
\acro{VSI}{variable \ac{Isp}}
\acro{RTG}{radioisotope thermal generator}
\acro{ASRG}{advanced Stirling radiosotope generator}
\acro{ARRM}{Asteroid Robotic Redirect Mission}
\acro{AATS}{Alternative Architecture Trade Study}
\acro{PPU}{power processing unit}
\acro{STM}{state transition matrix}
\acro{MTM}{maneuver transition matrix}
\acro{BCI}{body-centered inertial}
\acro{BCF}{body-centered fixed}
\acro{UTTR}{Utah Test and Training Range}
\acro{EPV}{equatorial projection of $\mathbf{v}_\infty$}
\acro{KBO}{Kuiper belt object}
\acro{DSM}{deep-space maneuver}
\acro{BPT}{body-probe-thrust}
\acro{4PL}{four parameter logistic}
\acro{BCF}{body-centered fixed}

\end{acronym}

% --------------------------------------------------------------------------------------------------------------------------
% --------------------------------------------------------------------------------------------------------------------------


%%%%%%%%%%%%%%%%%%%%%
\section{Introduction}
\label{sec:introduction}
%%%%%%%%%%%%%%%%%%%%%

There is currently only 1 test for PEATSA that has been created and included in the repo. The instructions to run this test will likely change rapidly as we create an expanded testing system for PEATSA. This first instance of a PEATSA test consists of 3 files providing a base trajectory and the required inputs to create a PEATSA run. THe purpose of this document is to outline the steps required to run this single test.

%%%%%%%%%%%%%%%%%%%%%
\section{Running PEATSA Test}
\label{sec:peatsa_test}
%%%%%%%%%%%%%%%%%%%%%

As of this writing, PEATSA is still only capable of running on Linux (not Windows). Accordingly, this first PEATSA test has not been built into Testatron and must be run on its own. To run the test, we must find the following 3 files in the EMTG repo (C:/emtg/testatron/peatsa\_test/):

\begin{enumerate}
	\item peatsa\_test.emtgopt
	\item peatsa\_test\_input.csv
	\item peatsa\_test\_input.py
\end{enumerate}

\noindent We must save a copy of these 3 files onto one of the Navlab servers where we wish to run the PEATSA test. Once placed, we must update the folder paths in peatsa\_test\_input.csv and peatsa\_test\_inputs.py (e.g. working\_directory and emtg\_rott\_directory in the .py file). Note that the hardware path and universe path for the EMTG input file will also need updating, but these can be done in the override\_options in peatsa\_test\_inputs.py.

\noindent When everything is updated to reflect the files' location, run PEATSA with the command:

nohup python /home/username/emtg/PyEMTG/PEATSA/PEATSA.py peatsa\_test\_inputs.py > terminal.out 2>\&1 \&

\noindent Note that this command must be run in the input file's directory. PEATSA will take less than a minute to run (stops automatically after Iteration2 finishes), and the results will be placed in a newly created peatsa\_test/ subdirectory. 

\noindent The test itself is conducted by the fourth script in the C:/emtg/testatron/peatsa\_test/ directory, PEATSATest.py. This script contains a function called PEATSATest which must be called. Either copy the output folder (peatsa\_test/) to your local machine or run Python on the server. The test can be run via the following Python lines:

from PEATSATest import PEATSATest

result, output = PEATSATest(`path/to/peatsa\_test\_inputs.py', `path/to/emtg/PyEMTG/')

\noindent Note that PEATSATest assumes that the output folder from the PEATSA run (peatsa\_test/) is located in the same folder as peatsa\_test\_inputs.py. The PEATSATest function returns a boolean result reporting whether or not the test passed. The output is a dataframe summarizing the results. If the test passed, result will be True and output will be an empty dataframe. 




\end{document}